\chapter{Sobre Macacos e Bananas}

Acordo. Olho o relógio. São 8:23 da manhã. Só mais cinco minutos. Acordo. Olho o relógio. São 8:28. Me levanto. Como algumas bananas acompanhadas com Co\-ca-co\-la Ze\-ro.

Estou seguindo essa dieta rigorosamente desde o início da semana, na tentativa de manter mais triptofano no meu cérebro durante o dia. Segundo Aldous Huxley, nossa mente só consegue receber uma certa quantidade de informação diária, devido a fatores evolutivos, desenvolvidos na época em que éramos mais macacos que homens. E um bom jeito de ``ver mais do que o normal'' seria através de drogas alucinógenas. Robert Anton Wilson recomenda a maconha, embora não seja por muitos considerada alucinógena, porém reconhece o maior poder de outras, como o \textsc{dmt}, a salvinorina, o \textsc{lsd}, a serotonina e o \textsc{lsa}.

Como eu não quero ``ver mais que o normal'' atrás de uma jaula, ver o sol nascer quadrado, finjo ser um macaco a comer bananas, que contém triptofano, precursor biológico da serotonina. Engraçado, este é um arquétipo símio bastante pobre, semelhante ao que temos dos coelhos, de que comem só cenouras. Se fosse eu um macaco, além de quase não comer bananas, já estaria certamente atrás de uma jaula. Não parecem funcionar, tanto as bananas, quanto as jaulas, para qualquer fim que seja.

Mas, sinceramente, não sei bem para que serve a Co\-ca-Co\-la Ze\-ro.

Depois de um banho, já estou pronto. Corro para pegar o ônibus. No ponto, olho o relógio. São 9:23 ainda. Espero cinco minutos e pego o primeiro \textsc{ufsc} semi-direto que encontro. Um carro da Transol, de número 0235. Não sei para onde vão essas latas azuis no final do dia mas, com certeza, vão todas para o mesmo lugar. Devem ser muitas e, como são todas iguais, não só é inteligente como indiscutivelmente necessário numerá-las. No fim, o ``0235'' serve justamente para pôr ordem naquele maldito lugar, seja lá onde for. Ônibus nessa cidade nunca foram rápidos, ainda mais que o meu destino é o terminal viário do centro.

Chego. Olho o relógio. São agora 9:55. Demorei, pensei. Vou ao chafariz do terminal. Aqui, no terminal municipal, existe um belo chafariz, como que para entreter os passantes com a água que, além de não a beberem nem a usarem para refrescarem-se --- certos estão eles, claro, afinal, não são animais, que nojeira seria! --- só a veem cair, cair e cair. Tentem fazer um chafariz em que a água só sobe, sobe e sobe. Ai então vão \emph{realmente} entreter os passantes.

Me sento em um banquinho, sorte de haver encontrado um. Um desses quadrados. Digo, cúbicos. Na verdade, cha\-mam-no de ``banquinho'' por consideração, carinho. É um bloco de concreto que, ao brotar do chão, revestiram com cerâmica, dessas bem baratas. Mas ainda assim serve muito bem para sentar.

Espero. Olho o relógio. 10:05 ainda. Logo chega uma garota, não muito alta, não muito baixa. De um ruivo que confunde. Será que é castanho? Não, na luz parece mais um loiro escuro. Estranho. Veste um, lógico, vestido. Fantástico. De cor indistinguível para mim, já que sou daltônico. Daltonismo é algo engraçado. Todos pensam que faz a pessoa ver em preto e branco, enquanto que na verdade apenas confundimos alguns tons, trocamos algumas cores. No meu caso, por exemplo, confundo alguns violetas por azul, alguns amarelos escuros por verdes e uns tantos azuis claros por tons de cinza. Com certeza ela observa cores melhores que eu. É que o daltonismo está relacionado com um gene recessivo no cromossomo masculino. Quer dizer que mulheres passam a desordem aos seus filhos, mas só a manifestam os homens. Discreto, sinceramente humilde o vestido, como de uma dama oriental. Mas ainda assim, digo, fantástico. Ouso dizer, digno de uma divindade seria. No específico caso, o é.

--- Vago está? --- pergunta a garota, agora sem poder ter a idade reconhecida. Que absurdo, seriam dezenove anos? Seriam trinta?

--- Claro, desde que não me arranque o sol --- respondi.

Sentou-se no ``banco'' atrás de mim. ``Atrás'' é discutível. Estes assentos são perfeitamente simétricos, senta-se na posição que agradar. No momento, estava eu sentado de costas para o sol, abraçando minhas pernas de leve. Ela então senta-se no banco ao lado, mas de frente para minhas ensolaradas costas.

--- O sol parado está, nem eu posso de lá o tirar.

--- Falo sobre as ondas eletromagnéticas que emana, não quero que teu corpo produza obstáculo para as pobrezinhas até meu corpo. Absurdo seria se com isso o sol se deslocasse. Haveria motivo para que pudesses fazê-lo?

--- Claro que sim. Digo, claro que não.

--- Entendo. --- Na verdade, porra nenhuma.

--- Somente digo. Remédios psiquiátricos, os tomo. Tente evitá-los. Não vão te bem fazer.

--- Quê? --- Me viro para olhá-la melhor, de espanto. Seria o português sua língua nativa? Pois não parece\ldots

--- Com essa tua dieta, haverias de várias dores de cabeça horríveis ter.

--- Dieta? --- Grandes olhos ela tem. Azuis ou cinzas? Talvez um verde muito claro. São como dois caleidoscópios.

--- Bananas. Queijo nesses tratamentos também deves evitar. Se não queres enxaquecas ter.

--- Mas eu não estou em nenhum tratamento desse tipo\ldots\linebreak E como sabes das bananas?

--- Não o sei. É tu quem sabes.

--- Quem és tu?

\begin{sloppypar}
--- Com certeza psiquiatra não sou. Sobre a guerra de Troia, já leste?
\end{sloppypar}

Onde está?

Acordo. Olho o relógio. São 8:23 da manhã. Só mais cinco minutos. Acordo. Olho o relógio. São 8:28. Me levanto. Como algumas bananas acompanhadas com Coca-Cola Zero\ldots
