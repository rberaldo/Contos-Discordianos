\chapter{Ode ao `quase'}

Uma pequena perturbação nas folhagens, um gemido do vento, um assovio das árvores. Quatro horas da manhã. Tarde assim da noite, os pássaros começam, timidamente, a cantar, como num murmúrio que se expande pelo profundo teatro, revelando então um coro \emph{a cappella} ao final.

Enquanto o tempo passa, ainda antes de o sol nascer, aque\-le astro que, ao aquecer as folhas faz vibrar o mundo e entra em gozo com os arbustos, como num clímax de homem e mulher, de céu e terra, os pássaros entram em frenesi. Algo está para acontecer, o grande segredo de tudo, este está para ser revelado. Está para ser relevado.

Se há um mistério digno de se revelar, excitante de se esconder, há um mistério que, ao se contar, tem-se o desejo saciado, como num depoimento, quando a testemunha, suando e aliviada, pode voltar para o calor do lar, do sol, ao ouvir, daquele que tem o torso adornado com as letras `s', `g' e `t', as palavras ``pode ir''. Vai. É o momento primordial. Se é mais noite a medida que o dia se aproxima, assim como se é mais humano a medida que o momento derradeiro chega. Se acaba, se é mortal. Se é.

O segredo é mais segredo segundos antes de ser contado, intervalo de tempo lembrado por quem o ouve e diz a si mesmo que agora o sabe, mas há pouco não sabia. Quando se é segredo no auge, tem-se que definhar, morrer, deixar-se contar, deixar de se esconder, deixar de ser segredo.

Os raios já vão aparecendo. O regozijo dos pássaros, que há instantes era colossal, agora mais parecem os gemidos de prazer em brasas, como de uma fogueira que quase se apaga: breve, profunda, mas ainda quente. Ah, como é linda a noite!

Já se é dia novamente.
