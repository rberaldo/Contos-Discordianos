\chapter{Amor, Sublime Amor\ldots}

Arthur entrou no costumeiro bar, sentou-se no lugar costumeiro do balcão; estava ansioso pelo jogo. Esperava fervorosamente que a Colômbia desbancasse o Brasil --- mas é claro que não diria isso para o povo ali, ou seria enxotado.

--- 'Noite, Hetz. Feliz dia do Nietzsche.

--- Boa noite, professor. Feliz dia do Bigode! --- E Hetzlinger riu bastante. --- E, ah, claro, feliz dia dos professores também. Veio assistir ao jogo, é?

--- Podes crer. Nem me lembre que fui professor, argh. Me vê o de sempre.

O bartender começou a preparar a bebida do velho Arthur, e, enquanto este esperava pelo jogo, João chegou, batendo-lhe nas costas e desejando feliz dia do Bigode. Enquanto isso, o Jornal Nacional rolava solto na \textsc{tv} do bar e uma notícia chamou a atenção de João:

--- Hoje, por volta das nove horas da noite, acabou o drama de Heloá, a jovem de quinze anos que estava sendo mantida como refém de seu ex-namorado, Lindembergue Fernandes Alves, de vinte e dois anos, inconformado com o fim do namoro\ldots\,--- o bartender abaixou o som da \textsc{tv} e comentou para Arthur e João:

--- Horrível isso aí, não? Um cara desses merecia\ldots

--- Merecia nada, cara. --- Arthur interrompeu Hetz, causando espanto neste e em João. --- É mais que natural que esse tipo de coisa aconteça. E, depois, não é ele que está fazendo isso, mas o gênio da espécie.

--- Xiiii, tu vais começar a viajar de novo, é, professor Schopenhauer? --- o bartender falou, e voltou a limpar copos, deixando o volume da \textsc{tv} alto novamente. João, porém, após pensar um pouco, acabou por concordar com Arthur.

--- Hetz, Arthur tá certo. Lembro de um trecho de \emph{EQM} que fala sobre isso.

--- \emph{EQM}? --- Arthur perguntou.

--- É, o livro do papa Ibrahim. Nunca leu?

--- Não, eu parei de ler.

--- Parou de ler, mas e a bebida? --- o velho filósofo fez um sinal de reprovação. --- De qualquer forma, eu me lembro bem do que o livro fala. É Falls, a namorada do protagonista, quem fala sobre a comunicação e como isto pode revelar como nos relacionamos com as pessoas. Olha só o que diz Falls:

\begin{quotation}
Voltando ao que eu queria dizer, esse pensador, Martin Buber falava de dois modos de expressão entre as pessoas. Ele falava da comunicação, entende? A comunicação se expressa de duas formas. EU-TU e EU-ISSO. O EU-ISSO é usado em nossas relações com o mundo das coisas. Essa é a minha casa. EU-ISSO, entende? O EU-TU são nossas relações com seres humanos. Eu quero passar o resto de minha vida com você. EU-TU.
\end{quotation}

--- Certo. Então você quer dizer que a relação entre Lindembergue e Heloá chegou à esta condição de eu-isso? --- Schopenhauer começou a pensar sobre isso. --- Realmente, faz sentido, mas, como eu disse ali em cima, isso tudo foi causado pelo gênio da espécie. Explico: o amor não existe; o que existe é a vontade da vida se perpetuar noutro ser, num terceiro ser, e isso é o que faz surgir a admiração por alguém. Explicar tooodo o mecanismo seria um tanto enfadonho agora, portanto, fico por aqui, só para apresentar minha conclusão: Lindembergue não merece nenhum tipo de punição, apesar dos pais deles merecerem pelo péssimo gosto ao escolher o nome do filho; ele está sendo movido por algo que é maior que ele, maior que todos os desejos dele. Afinal, o que pintam todos os poetas e depois chamam de amor é, tão somente, algo que é transcendente a eles e, portanto, não podem entender em sua plenitude. Porém, o amor acaba quando ele realiza seu desejo: criar um terceiro indivíduo.

--- Schop, te digo mais: Nietzsche concordaria contigo quan\-to a isso de não punir o cara --- exclamou Hetz --- já que o Bigode disse, em \emph{Assim Falou Zaratustra}:

\pagebreak
\begin{verse}
Sempre se viu só, como o autor de um \\
ato. Eu considero isso loucura; a \\
exceção converteu-se para ele em \\
regra. \\
\end{verse}

Então, os três se olharam, olharam para a \textsc{tv}, e Schopenhauer disse:

--- Muito bom, gente, mas, agora, eu tô afim é de assistir o jogo. Tá pra começar.
