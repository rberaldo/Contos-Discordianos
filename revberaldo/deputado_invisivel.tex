\chapter[Romão Brasil, o Deputado Invisível]{Romão Brasil, o Deputado Invisível, Justo como Nenhum Outro nesta Nação}

Romão Brasil, em seu melhor terno, na sua luta pela sociedade e pelas crianças. Entra no carro oficial; para a escola de minha filha!, ordena ao motorista que, sorridente, completa seu trabalho a tempo.

Hoje é o Grande Dia de inspirar crianças.

Na escola da filha, seu aguardado discurso:

--- Crianças, quando era do tamanho de vocês, diz aos pequenos de no máximo sete anos, não gostava de política. Mas quão contraditória é a vida!, pequenos! Ele olha orgulhoso para o futuro da nação, que neste momento está absorta em ``desenho livre''. Você!, continua, aponta vagamente para um garoto, vocês! são o futuro e a esperança da nação. Contarei uma pequena história. Quando eu cresci o suficiente para entender o mundo, como vocês farão em breve, e deu um largo e ufano olhar para sua filha, comecei a me interessar pelo mundo da política. Fazia pequenos comentários, inflamados e cheios de esperança, entre meus colegas d'escola. As crianças continuavam a desenhar, e João Brasil, justo, ignora e continua. Cresci como um garoto de bem, me formei em direito, e tudo fomentava meu futuro e gôsto pela política. Nem mesmo a professora olha para ele; passa folhas pelo mimeógrafo, imprimindo nas mentes jovens o cheiro do álcool em suas lembranças infantis. Porém nunca pensei em me candidatar, até conhecer o sr.~****, quando eu já exercia a profissão de leis há dez anos, que me fez entrar para a política efetivamente, honrando o espírito de cidadania e serviçalismo pelo povo que sempre senti.

Não entendia. Ninguém parecia ter ouvido uma única palavra de seu discurso. Sua filha, naturalmente, ávida, composturada, espera por mais. Que garota, que orgulho para um pai! Certamente dará boa juíza, \emph{desembargadora}, como ele não foi. De espírito renovado, continua João Brasil:

--- Ó! Juventude! Se soubessem o que estão perdendo\ldots\,A professora levanta e começa a escrever no quadro. Brasil se sente ultrajado; tenta chamar a atenção da professora que, impassível, ignora-o como se fosse \emph{transparente}.

--- Pai, diz sua filha, vamos embora, ninguém te ouve aqui.

Um suado João Brasil acorda em banho maria na cama d'hotel. Que pesadelo, por Deus, que pesadelo! Não ser mais ouvido\ldots

\begin{center}
{\huge $\rightarrow\leftarrow$}
\end{center}


No carro para o Senado ou para a Câmara ou para qualquer lugar onde legisladores legislam. Romão Brasil coça o nariz e abre sua agenda. Muitas coisas a serem feitas. Será um grande dia. Na sua pasta impecável, translúcida, um impresso guarda seu discurso. O Discurso que Irá Mudar o Brasil, como gosta de chamar em seus pensamentos, rindo jovial do jogo de palavras. Conta com a aprovação imediata dos colegas do partido. Conta com a aprovação imediata do Sr.~Presidente da República. Conta com a aprovação imediata até mesmo da oposição. Conta com a aprovação imediata do povo brasileiro.

``Romão Brasil'' estaria escrito em pedra em todos os corredores da glória. Mas, antes!, antes!, antes o povo. Pensar em glória só fará com que nada dê certo. Antes o povo! que minha glória, adagia Romão Brasil.

E assim Romão marcha pelos corredores cumprimentando, derramando gotas de sinceridade e luz matinal em cada um de seus compatriotas. Mas\ldots\,Estranho!, nenhum deles responde, fossem da oposição, filiados, companheiros de partido. Estranho!, estranho!, mas compreensível, pensa a brilhante mente do político. Sinceridade nunca é apreciada!, sinceridade radiante, superação intelectual e moral. Nunca! Estão com inveja, todos. Só reforça minhas expectativas!

Sobe no palanque. Câmeras transmitem em cadeia nacional o momento bombástico. Memorável. Revolucionário. O discurso jorra em fontes de vitória pelos alto-falantes. Mas lentamente, sutilmente, Romão nota que a sua plateia dormita, sonha acordada, joga sudoku.

--- Senhores, isso é um ultraje! Romão resolve incrementar a revolução, estendendo-a ao momento. Um ultraje! Meus colegas, senhores, aqueles que podem mudar o país\ldots\,Os representantes do povo, encarregados de avaliar propostas que representem este mesmo povo\ldots Estou indignado. Tantos anos de trabalho, servindo a nação, e nunca encontrei um grupo de\ldots\,De deputados tão\ldots

Antes que possa acabar, Romão Brasil é surpreendido por um deputado que se aproxima de seu lugar. Entende que havia acabado seu tempo. Entende que o sonho de melhorar o Brasil estava amarrado duramente à burocracia de procedimentos. Continua seu discurso --- não poderia deixar isso acontecer, afinal! --- mas seu companheiro de trabalho simplesmente começa a falar.

Antes que pudesse interromper, a verdade, de tamancos, faz uma visita e Romão Brasil desmaia.

\begin{center}
{\huge $\rightarrow\leftarrow$}
\end{center}


Romão vai ao psicólogo.

--- Doutor. O que tem acontecido é completamente estranho. Parece que as pessoas não ouvem, não me veem mais!, terrível pensamento para um homem que escolheu vias tão certas numa profissão permeada por tentações. Sou um dos poucos políticos, se me permite a falta de humildade, sou um dos poucos políticos corretos deste país. E agora parece que ninguém mais pode me ver ou ouvir! Tudo começou com uma série de sonhos assustadores e agora isso passou para a realidade. Ou, já nem sei se isto é mais realidade, sonho, ou se aquilo foi sonho alguma vez. Romão sua frio. Quanto mais fala, mais sua. O psicólogo está absorto em anotações, rabiscos em seu moleskine.

A porta abre. A secretária entra. O psicólogo mostra seus rabiscos a ela. O deputado continua a falar.

--- Belo desenho, doutor.

--- Muito obrigado, Gabriela. Por favor, chame o próximo paciente.

Sapatos fazem barulho. A porta --- se fecha.
