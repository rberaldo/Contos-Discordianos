\chapter[Nos Bastidores da \textsc{tv} e da Religião]{Enquanto Isso, nos Bastidores da \textsc{tv} \& da Religião\ldots}

Brasília. Oásis de brilho parco, concrético. Las Vegas brasileira.

Bancas de jornais ardiam às três da tarde. O sol evaporava violentamente o espelho d'água. Faxinavam um corredor de Nova Versalhes. Carros jorravam $\textrm{CO}_{2}$ no deserto. Um fardado de terno olhava a cidade pela persiana da janela ministerial. O celular tocou. Tirou-o do bolso, abriu, deixou que a sombra cobrisse seus olhos, e atendeu. Alô?

``Ah, consegui te ligar. Estou indo praí, e o sinal estava péssimo. Deixe-me explicar o que está acontecendo, posso?''

``Vá em frente.''

``Ótimo. São muitas coisas. Menciono algumas: o \textsc{ibope} da Record. Você sabe, eles nos ultrapassaram. Isso não parecia ser um grande sinal, na realidade\ldots\,E algumas pessoas daqui me disseram que talvez fosse melhor se manter na retaguarda e não correr riscos ainda. Mas quem manda pensa diferente. Além disso, há um projeto de lei aí. Noticiaram isso hoje, eu digo, mas ninguém sabe muito bem o que pode virar. É algo sobre mais poderes religiosos, não sei bem. É um dos motivos pelo qual estou indo aí. Por outro lado, tem esse acordo com o Vaticano\ldots''

``Vocês não estão indo longe demais?''

``Não, quero dizer, o Vaticano não vai nos comprar. Não estou sugerindo nada sobre o Vaticano. Mas o número de crentes está enorme, você sabe. Esse Edir Macedo tem todos esses rolos que estamos soltando, mas isso precisa de uma ajuda de vocês também. Nós estamos tentando libertar a população do feitiço\ldots\,Olhe, há quem se vicie em pagar o dízimo. Então viciam no canal da igreja. Estou indo direto demais ao ponto?''

``Estou acostumado.''

``Ah, sim. Bom, eu nunca fiz isso, mas me pediram pra ser sincero. Ou inventar alguma sinceridade que pareça ser de utilidade pública. Bom, vocês são a utilidade pública, e nós temos o poder pra divulgar\ldots\,Ainda. Quero deixar claro: não é algum tipo de lobby. Mas isso simplesmente não pode chegar a um nível que vire uma jihad --- e você sabe que há quem goste da jihad aqui.''

O fardado suspirou. ``Eu sei.''

``Perfeito. Há uma janta marcada. Estou te convidando. Vo\-cê pode ir?''

``\ldots\,não sei. Podemos conversar quando você chegar?''

``Sim, claro, mas pense que essa janta pode ser útil pra vocês e para nós, e podemos torná-la boa para você e para mim.''

O fardado parou pra pensar. O cara que liga é meio novo. Mas ele vem com instruções do que falar, e isso não é pouco. Disseram para esperar pela ligação, e quem disse não é gente de se ignorar. Tencionar um pouco pro lado do garoto não fará mal.

``Vamos deixar isso meio certo. Preciso verificar agenda e essas coisas. Falamos quando chegar.''

``Me disseram que vão me largar na frente do museu, na Praça dos Três Poderes. Dê uma passada ali, eu te ligo quando estiver lá. Tente não trazer mais gente, não estou acostumado com isso.''

``Perfeito. Bom resto de viagem. Até mais, abraços.''

O cara que ligou se despediu.

Aquilo era sério.
