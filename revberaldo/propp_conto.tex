\chapter[Um conto em três actóides]{Um conto em três actóides (que também serve como exercício proppiano da faculdade)}

Ato um: era uma vez. Muito feliz. Ela pediu: vá, minha filha, Joana. Mas\ldots\,Me promete? Prometo, mãe. E foi. Como muito jovem, ignorou o mas anterior. E continuou indo. Então parou. Um môço. Antônio E., nome dele (lobo, esse lobo do homem). Família nobre, Moraes. Sem morais. Sem mais: diálogo: ?, ., ?, ., ?, \ldots\,Sim, caminho melhor, esse, Joana. E Antônio E. foi. Joana logo também. Voltas, voltas, voltas, demorou, chegou. Entrou, aberta. Gritos. Vó?

\begin{center}
\emph{(Fecha o cenário mental, abre outro. Terras longínquas, aparece Siegfried. De escopeta, não espada.)}
\end{center}

Ato dous: com cabeça de dragão inda rolando, passa Siegfried, herói da gente deles, e ouve gritos. Qué qu'houve? Ele só ouve. E se'proxima. Por Jesuis Shiva Thor!, comeu a vó e a criança, maldito seja! E tenta se aproximar, nosso herói, mas encontra os seguranças. Bang, bang, BANG, isto é, tiros. Carnificina. Lobo Antônio foge. Helicóptero. Aparece a Valkyrja, die walküre, e cena de perseguição. Narrador: ``eis que na avenida perseguição inusitada, três carros de polícia atrás de cavalo e helicóptero''. Mundo lóki. 'Caba a gasolina (especial, evidentemente) do aeromóvel a garota do cavalo sorri pára Siegfried o herói saca'escopeta e mete três tiros nos cornos do lobo ufa.

\begin{center}
\emph{(Por sorte, Joana e sua avó saem vivas da barriga do homem de taras estranhas que as comeu. Metabolismo lento, saravá.)}
\end{center}

Ato \foreignlanguage{english}{three}: eis que'á Justiça chega, heroína da nossa gente. Salvadora, de papel timbrado, a enxugar o sangue derramado. Fala-se, não se mata um homem assim, não. Por mais bizarras as taras. Não: sociedade, julgamento certo ele merecia. Sem pedras \& paus, com papéis e fala, Siegfried é perseguido, porcos atrás dele. Dá uns tiros, monta no cavalo da Valkyrja, dispara com vovó e Joana atrás. Deixa vovó em casa, com cesto de doces e curadíssima. Voltando pra casa de Joana. Mamãe é muito sozinha, e mostra uma foto. Herói solitário, se interessa. Mas a Justiça os pega. Siegfried em cana, Joana em casa, felicidade geral, sai no jornal. Julgamento: demora. Siegfried espera, argumenta, acha bom advogado. ``Era um demônio, aquele lobómem''. Júri divido, no fim concorda (martelinho batendo; veredicto). ``Um demônio'', agora a opinião pública convencida recita. Vilão morto, dinheiro doado prálguém. Siegfried casado, com a mãe. Joana claustrofóbica pra sempre; ``vai virar lésbica'', vaticínio freudiano.

\begin{center}
\emph{(E foram felizes para sempre. Um figurante olha para o leitor e diz que ``o mediador entre a cabeça e a mão deve ser o coração'', aplausos.)}
\end{center}
