\chapter{Retirado de um velho pedaço de jornal}

{\Large \textsc{mais que autoajuda, escritor promove o autoconhecimento}}

São Paulo --- Ontem (23) ocorreu o lançamento do livro \emph{O hippie que não era sujo}, de Astolfo Voltaire, pseudônimo de Alberto J. Goldsmith-Paes, o internacionalmente famoso multimilionário brasileiro. O livro, custeado por ele mesmo e editado pela ed. 3K, foi lançado ``chiquemente'' no distinto hotel de São Paulo, com banda contratada e distribuição de exemplares entre os presentes.

A equipe deste Jornal conseguiu uma descontraída entrevista de vinte minutos com Goldsmith-Paes.

\emph{Pergunta --- Como foi que você resolveu lançar o livro} O hippie que não era sujo\emph{? Comente um pouco sobre seu conteúdo.}

\textbf{Goldsmith-Paes ---} Resolvi lançar o livro quando notei que existia um conflito muito gran\-de nas pessoas, principalmente entre os jovens, entre ganhar dinheiro ou carregar nobres ideais como amor, sinceridade, compaixão, empatia, altruísmo. Muita gente vinha e me perguntava, ``Alberto, mas como você consegue conciliar essas coisas? E a exploração etc.?'' Eu realmente não tinha parado para pensar nessas questões até então. Mas pensei e achei algumas soluções. Quanto ao conteúdo, ele se relaciona diretamente com o título: um \emph{hippie} em plenos anos sessenta que deseja seguir os passos do pai e se tornar bem-sucedido, mas ao mesmo tempo se identifica com os ideais de seu tempo.

\emph{P --- E a escolha dessa personagem foi feita por qual motivo?}

\textbf{G-P ---} Creio que esses belíssimos sentimentos de que trato no livro são rapidamente relacionados ao movimento \emph{hippie}. É o acontecimento do tipo mais conhecido e temporalmente perto de nós. Esse foi o motivo pelo qual escolhi um \emph{hippie de coração}: ele vivia num momento de tensão máxima, então representa bem o conflito pelo qual passam algumas pessoas.

\emph{P --- Algumas pessoas classificaram seu livro como de \emph{auto\-ajuda}. Isso te incomoda? Como você o classificaria?}

\textbf{G-P ---} A última pergunta primeiro. Eu prefiro não clas\-si\-ficá-lo. Acredito que seja do tipo de coisa nova, mas sempre há quem acabe por classificar. Por outro lado, chamá-lo de \emph{autoajuda} não é completamente ruim, só não capta tudo o que o livro traz. Ele é muito mais que autoajuda, já que toda a ajuda que a pessoa precisa está nela e não no livro. \emph{O hippie que não era sujo} apenas guia as pessoas por esse caminho que é a mente delas, suas ideias.

\emph{P --- O público esperado é majoritariamente jovem?}

\textbf{G-P ---} Sim. Mas é um livro para todos, e não só aqueles que desejam ser financeiramente bem-sucedidos. Ele tenta conciliar as contradições do mundo em que vivemos e, eu acredito, é muito eficiente em atar as pontas. O livro é uma realização, de todo modo.

Alberto Godsmith-Paes é também conhecido por escrever o ``livro-cartilha'' \emph{Cinco maneiras para não reagir a um assalto}, onde alterna o tema explícito no título com bom humor, contando experiências suas e de conhecidos, e também o livro \emph{O que os golfinhos não podem fazer}, onde faz uma análise otimista das conquistas e rumos da humanidade.
