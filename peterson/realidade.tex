\chapter{Realidade..}

Ela estava sozinha em casa. Ela gostava de ficar sozinha em casa, era quando podia usar pijama o dia inteiro, almoçar às duas da tarde e ficar mais tempo no computador, fazendo suas coisas úteis e inúteis.

O que incomodava a criança crescida, que já não tinha medo do escuro, eram as coisas que ela sentia de noite quando dormia sozinha. O quarto tinha uma luz um pouco fraca, e a luz da tela do computador era tão fantasmagórica. O corredor escuro e longo parecia um túnel do tempo, ambientado pelo tic-tac do relógio da cozinha.

Ela ouvia ``Time'', do Pink Floyd, o solo enorme e precioso, a música lisérgica. A qualidade lhe excitava, mas a categoria a fazia sentir sono. Fazia muito frio naquela noite, e o vento assobiava quando encostava nas entradas ``ilegais'' da janela. Como já era estranho o suficiente ficar com todas as luzes da casa apagadas, ela não quis ficar olhando para o mato grande no fundo da casa. Ali havia muitos bichos estranhos, insetos, etc; ela não tinha medo de bichos, amava os bichos. Mas não gostaria de acordar olhando para, sei lá, um tatu gigante na janela. É claro que o fato da existência ou não de um tatu gigante é totalmente discutível, mas a julgar pelas características já vistas em alguns animais estranhos dali, não seria de se espantar se ela visse um tatu gigante quando acordasse.

A música agora é The Greag Gig In The Sky. Sonzinho triste; ninguém online no \textsc{msn}. Foi escovar os dentes, tarefa tão adiada. Passou pelo corredor longo e escuro, com o barulho do tic-tac misteriosamente parecendo mais alto --- ah, essas armadilhas da mente\ldots A música já quase inaudível. À noite, o silêncio tem um som próprio. É quase um zumbido, só que não muito forte; uma coisa que só podemos ouvir no silêncio absoluto e que nega o próprio silêncio absoluto --- o único que realmente convive com o silêncio absoluto é o surdo. Todos os outros podem ouvir essa pequena freqüência que vem sei lá de onde e chega sei lá como nas nossas orelhas.

Pronto, a mulher parou de berrar. Não, ela continua. Essa música é meio chata nesse sentido. Ela queria guitarra, o piano e o vocal não eram suficientes. Viu um vulto no espelho, se assustou um pouco. Quê isso, era só uma camisa preta pendurada. Pensou ter visto um cabelo por um instante, daquels grossos, como os da Samara.

Enquanto escovava os dentes, ficou olhando fundo nos seus olhos pretos. ``Eu ficaria bonita morta?'', ela pensou. Que pergunta tola\ldots

Começou a lembrar-se dos filmes que viu. O Chamado, O Grito, Visões. E os outros, mais clichês. Ou pelo menos clichês mal-feitos.

Ela se lembrou de como sentiu medo quando viu O Chamado pela primeira vez. Os olhos azuis intensos daquela menina tão má e terrível, o vídeo bizarro, que quando ficava em tela cheia parecia estar sendo realmente visto pelo espectador; aquela mosca, o olhar penetrante da mulher penteando o cabelo\ldots O ato de pentear o cabelo é fúnebre, pensou a garota, e começou a imitar a mulher, a tal mãe da Samara. Sentiu um calafrio percorrer seu corpo. Guardou a escova com supersticiosa apreensão, engolindo em seco mas sabendo que, afinal, isso não era nada.

Ela podia ouvir o refrão catártico de Us and Them, toda a grande glória musical da música e quiçá do álbum inteiro ali concentrados.

Ela se lembrou de novo da Samara. Apagou a luz branca do banheiro e, quando voltou, voltou de costas, olhando para a escuridão da cozinha, que mais lhe sugeria uma grande passagem para outro mundo, uma grande ponte para a morte, isso sim. Ó, o refrão se aproxima. Agora. Isso\ldots Lindo. E aterrorizante, sem dúvida.

Fechou a porta do quarto. Por quê? Pff\ldots Nada poderia acontecer. Pausou a música. Agora um avião passava por cima da casa, muito, muito acima dela; um cachorro latiu uma ou duas vezes. Mas uma coisa permaneceu: o barulho da noite.

O barulho do silêncio, a estática que persegue todos os que têm ouvidos, condenando-os a jamais deixar de prestar atenção a esse detalhe cinzento, nublado, estranho. O silêncio e seu barulho característico parecia um grande e sinistro prelúdio. Sutil, porém grandioso\ldots

Desistiu de ouvir o negro agouro; apertou o play de novo. Que era aquilo, ela não era mais criança pra ter mais desses pensamentos bobos. Fechou os olhos, cansada. Ficou olhando para a porta fechada. O saxofone a relaxava, mas o vento na janela estava ficando tão mais forte\ldots A maçaneta tremeu um pouco. Ela ficou imaginando o que aconteceria se alguém abrisse a porta. O saxofone e o backing vocal do refrão, lindo, lindo\ldots Ficou imaginando a mulher sem queixo de ``O Grito'' abrindo a porta; a mão ofídica, branca, morta, unhas podres, abrindo lentamente a barreira marrom. Mas ela não conseguia imaginar o terror de se encontrar com esse tipo de sobrenatural por muito tempo; o filme de sua mente chegava até a parte em que os olhos dela passavam das mãos da praga até os olhos, sempre vermelhos, intensos, terríveis, avassaladores\ldots Ela parou de pensar nisso. A música não colaborava. Any Colour You Like, acabou de começar. Quer saber? Chega de Pink Floyd.

O Shuffle sacaneou. Radiohead.

O violão de How To Disappear Completely já começou. Aquele barulho estranho e bizarro já começou também. Aquele som de fundo, que é basicamente o som do silêncio e da noite ampliados, faziam ela se perguntar por que ela continuava ouvindo aquilo. Fechou os olhos. Não conseguiu. Por que diabos estava tão frio? Ela se levantou, se ajeitou na cadeira, passou a mão no nariz. Olhou pra porta. Olhou mesmo. Olhou com olhos cautelosos.

Por diversão ou por indução, voltou a imaginar situações as mais diversas possíveis, onde monstros de filmes de terror --- em geral espírios maus --- invadiam o quarto. O esterótipo é bem comum: a roupa nojenta e pútrida, rasgada; o cheiro não importava, não haviam cheiros no filme pra compor o estereótipo; os cabelos negros escorridos e tão nojentos e repulsivos quanto o corpo todo; os olhos, sempre os olhos, e um sorriso macabro no rosto. Um sorriso que dizia ``Eu vim. Estou aqui''. Uma risada fraca, mas que fazia alguém tremer de medo, dos pés à cabeça. Um rosto branco, veias à mostra, e finalmente o contato. A mão que se estende em direção ao rosto. A aproximação, o que é para os românticos do fim da vida, o grande símbolo da tragédia completa.

Era tudo tão vívido, uma cena tão real, tudo tão real. ``I'm not here. This isn’t happening''. Estava quase chegando na parte em que era possível sentir o vento passar pelo seu rosto quando você se imaginasse caindo de um prédio. Engraçado como ela podia ver cada nuance do rosto terrível e degradante do monstro humano feminino que se alojou em seu quarto. A porta estava aberta, e ela conseguia imaginar até mesmo ela gritando, se contorcendo tentando evitar o fim\ldots Era agora\ldots

Enquanto fechou os olhos para sentir a música, soltou uma risadinha. ``Sabe o que é a realidade? Aquilo em que, quando você deixa de acreditar, não deixa de existir''.

E quando ela abriu os olhos, o defunto ria. Ele não deixou de existir.
