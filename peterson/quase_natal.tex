\chapter{Quase Natal}

\begin{flushright}
\emph{Dedicado a Juliana Pires}
\end{flushright}

Era uma madrugada quente de janeiro. Sentado na varanda da casa rústica no sul do estado, ouvindo o barulho típico dos grilos e outros bichos esquisitos, estava o casal. Nenhuma situação muito romântica; era apenas\ldots\,Neutra.

Ela observava o milharal à frente da casa, no terreno do pai do namorado. Ele ora lançava olhares furtivos ao corpo dela, ora lançava olhares poéticos para o céu. No primeiro caso, não sentia nada de específico ao fazê-lo; era pura contemplação, momento tão raro. Gostava de olhar por olhar, e fazer os olhos irem e virem na perna destapada pelo short jeans, como que lendo uma palavra grande e confusa várias vezes, ou mesmo olhando pra um aquário muito extenso, num espaço de tempo lento e sem som algum. No segundo caso, sentia a esperança boba de achar uma estrela cadente. É claro que isso não aconteceria. Nunca tinha visto uma, sempre achou péssimos os efeitos especiais de filmes que simulavam uma e tachava (apenas pra si mesmo, numa nota mental) de mentiroso qualquer um que dissesse que havia visto uma.

--- Milho não te dá medo?

--- Quê? --- disse ele, rindo, acordando de suas ilusões bobo-contemplativas.

--- O milho é quase um personagem principal das histórias de terror de fazendas. Se tem um monstro ou bicho ou espírito, e a história é numa fazenda, a fazenda vai ser de milho!

--- Mas é porque milho é só o que eles plantam nos Estados Unidos.

--- Quê? Claro que não! --- disse ela, cética e risonha.

--- Claro que é! Por isso que em qualquer filme o milho é o personagem principal. Milho é tipo o Capitão América pros americanos.

--- Então não é nada, né? Porque o Capitão América é muito chato.

--- Chato? --- ria ele, fazendo ela rir em resposta --- Por que chato?

--- Ah, para. Você já perguntou pra alguém \emph{ei, e aí, qual é o teu super-herói preferido}? E aí te responderam \emph{Capitão América!} \ ?

\begin{sloppypar}
--- Não, mas aqui ninguém dá a mínima pra ele mesmo\ldots\,Mas eu nunca perguntei isso pra um americano!
\end{sloppypar}

--- Que americano seria idiota pra gostar de Capitão América?

--- Por que essa implicância com o Capitão América?

--- Porque ele é ridículo, ele é tosco! É como a gente gostando do Zé Carioca porque ele é o brasileiro que foi morar na Disney! Foda-se o Zé Carioca, ele é muito chato!

--- Olha, na verdade\ldots\,Nem tem diferença! Todos os heróis são muito parecidos, e os personagens de desenho animado também são\ldots

--- Ah, \emph{não são não}\ldots

--- Claro que são! Quer ver? Pega uma história do Zé Carioca.

--- Hum.

--- Agora tira o papagaio e bota o Pato Donald ali. Duvido que você não vá se matar de rir.

--- \emph{Nada} a ver! Se for o Pato Donald eu vou ficar imaginando ele falando com aquela voz dele, e \emph{aí} eu vou rir.

--- Você \emph{imagina} a voz deles no gibi?

Ela olhou pra ele com uma cara que misturava decepção e sarcasmo.

--- Que foi? --- perguntou ele.

--- Tá falando sério?

--- Sim, ué. Ah, é que você sabe, eu leio gibis passando muito rápido, todos os personagens têm meio que\ldots\,Vozes parecidas, de acordo com o estereótipo.

--- Bem coisa de quem não lê HQ mesmo --- e ela voltou a ficar de frente pra plantação, olhando o vento fraco mexer uma ou outra planta.

Depois de uma pausa que não foi desconfortável porque ele já estava gastando seu tempo rindo, por dentro, da conversa com um fim potencialmente desconfortável, ele resolveu retomar a conversa:

--- Filmes japoneses ou outras coisas assim de terror não têm milho.

--- Filmes orientais de terror são \emph{urbanos}, se passam na \emph{cidade}. É óbvio que não têm milho.

--- Mas se passassem na fazenda, não teriam milho também. Isso é coisa de americano.

--- Amor\ldots\,Têm fazendas no Japão? --- perguntou ela. Os dois começaram a rir com sorrisos abertos, sinceros, ainda que não gargalhassem; olhavam um para o outro e alguma coisa forçava a boca deles a continuar aberta. Ela tinha bebido um pouco e era um pouco visível que ela não estava muito no controle dos músculos faciais.

--- Uma vez --- começou ele, ficando de lado na cadeira pra olhar reto pra ela --- eu ouvi uma história, uma lenda aqui da região, sobre um assassinato.

--- Como é --- ela, um pouco mais desajeitada, foi rolando até ficar de bruços.

--- Assim você me desconcentra --- comentou ele, fazendo ela rir. Desconcentrava mesmo.

--- Desculpa, mas agora eu tô a fim de ouvir a história --- ela virou de lado, interessada --- concentra.

--- Tá, pra resumir\ldots

--- \emph{Não}! --- os dois riam da safadeza --- Se não for a versão sem cortes, eu vou cortar outra coisa na nossa programação, querido\ldots

--- Tá, tá\ldots\,Bem, era uma vez um homem..

--- Não, pera\ldots\,Era uma vez?

--- Tá, mas\ldots\,Como é que você quer que eu comece?

--- De um jeito criativo. \emph{Surpreenda-me} --- ela estreitou os olhos pra ele, que pôs a cabeça pra funcionar. Não querendo se estender muito, falou a primeira coisa que surgiu à sua mente:

--- Era uma noite quente de verão\ldots

--- Pronto, agora vai --- ela estava séria. Não gostava dos risinhos típicos que quebravam o clima das histórias de terror.

--- Ok. Era uma noite quente de verão, e a cidade estava apinhada de turistas. Todos vieram atrás das odiosas casas baratas que você consegue encontrar ali na praia da Teresete.

--- A gente conhece o tipinho.

--- É. Então\ldots\,Isso foi em Dezembro ainda, segundo o que me disseram. Era quase natal, e as lojas faziam aquelas promoções típicas. Mas não havia nada de típico quanto ao vendedor de uma das lojas, que naquela noite resolveu cometer um \emph{assassinato}.

--- E foi matar quem?

--- O próprio irmão\ldots

--- Uau\ldots\,Por quê?

--- Ninguém sabe; na época todos disseram que eles tinham uma \emph{ótima} relação, nunca foram vistos brigando feio, etc etc etc. Ninguém conseguiu entender o motivo.

--- Então pegaram ele mesmo?

--- Não, não! O legal da lenda é que não pegaram, se pegassem teriam descoberto o motivo.

--- Ou não\ldots\,Vai, continua.

--- Bem\ldots\,O caso é que a lenda é justamente uma lenda porque é a explicação que há por aí pro desaparecimento desse irmão do vendedor. O vendedor se chamava Renato, e o irmão, Odílio.

--- Odílio foi o que desapareceu?

--- Sim. Do dia pra noite, ou melhor; da noite pro dia. Sem deixar rastros. A última vez que foi visto foi entrando na loja em que o irmão, o Renato, nosso assassino hipotético, estava trabalhando até tarde da noite, arrumando o estoque, essas coisas.

--- E aí ele matou o irmão lá dentro da loja?

--- Sim. Dizem que foi com uma faca; ninguém ouviu gritos, ninguém ouviu tiros, ninguém ouviu nada. Mas fato é que Odílio nunca mais saiu de lá com vida.

--- Mas então ele \emph{saiu} de lá?

--- Ah, saiu. Saiu, ficou na frente da loja por uma ou duas semanas e então foi embora de novo\ldots

--- C-como assim?

--- Aí começa a lenda. Quer dizer, já começou antes, mas aí vem a parte boa dela. O que dizem é que Renato viu o que tinha feito com o irmão e agora não sabia o que fazer com o corpo. Foi aí que ele viu o Papai Noel gigante que a loja tinha comprado pra colocar na frente da loja, sabe, essas coisas que os lojistas acham que vai atrair mais compradores e coisa e tal\ldots

--- Sei, sei.

--- Então: o Papai Noel era tipo \foreignlanguage{english}{\emph{actual size}}, sabe? Cabe uma \emph{pessoa} ali dentro.

--- Ah, \emph{tá brincando}!

--- Não\ldots\,Renato fez um corte nas costas dele, que iam ficar viradas pra uma poltrona de qualquer jeito, tirou o enchimento do boneco, \emph{jogou} o corpo do Odílio ali dentro e costurou tudo de novo.

--- Caramba\ldots

--- Sim. Aí limpou todo o sangue, se livrou das provas maiores e lá ficou o Papai Noel, na frente da loja, dia após dia\ldots\,Depois que acabou o Natal, ele mesmo se encarregou de tirar a decoração. E \emph{enterrar} o Papai Noel.

--- Nossa\ldots\,--- A história teve um efeito que nenhum deles, tão acostumados a ouvir e a contar essas histórias, esperavam. O ambiente ficou mais pesado; o quente e opressor ar parecia não mais fazer daquela noite uma noite agradável, e tudo ali fora meio que sugeria \emph{perigo}. O quase silêncio (malditos grilos) também não ajudava.

--- Eu acho que eu vou dormir\ldots\,--- comentou a garota, no mesmo instante em que uma pick-up estacionava em frente à propriedade. Era Márcio, um amigo que tinha vindo junto passar o feriadão por ali.

--- E aí Márcio\ldots\,Cadê o Joel? --- Joel era o irmão dele.

--- Foi, ah\ldots\,Dormiu. Na casa de uma amiga.

--- Que amiga?

--- Acho que ele conheceu hoje --- disse ele, dando uma risadinha que se estendeu ao longo da frase.

--- O que é aquilo ali dentro do carro?

--- Isso aqui? --- perguntou ele, apontando com o polegar pro lado; no espaço atrás dos bancos da frente havia algo parecido com uma pessoa ali, vestindo uma roupa vermelha; os vidros estavam um pouco sujos, então não dava pra ver direito --- É só, é\ldots\,Uma fantasia que eu aluguei pra festa a fantasia da terça.

--- Ah, legal. Do que você vai?

--- P-Papai Noel --- respondeu ele --- Mas não sei, talvez eu.. Vou alugar outra ainda. Uma melhor.
