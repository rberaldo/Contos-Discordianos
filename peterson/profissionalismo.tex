\chapter{Profissionalismo}

Maurício era uma pessoa comum. Tomava café, comia carne, visitava o Orkut. É claro que isso não o tornava medíocre; não há nada de errado em ser uma pessoa comum.

Maurício tem até mesmo um trabalho. Um trabalho que, hoje ele vê, não tem nada de muito especial. Hoje ele vê seus horários de trabalho como obrigações cinzentas, meros processos, meras rotinas pelas quais ele têm de passar. Nunca a etimologia da palavra lhe foi tão verdadeira; o \emph{tripalium} podia não maltratá-lo violentamente, mas afinal, as coisas que irritam são piores que as que fazem sofrer.

Nunca foi uma vida extremamente aprazível, vá lá, sua infância foi ok, mas \emph{oh lord}, nunca foi tampouco assim, \selectlanguage{english}{\emph{shitty}}. Ele olha para a sala mal iluminada --- coisa chique, coisa chique --- com um pouco de desprezo. O quadro de flores da tia, que não lhe desperta emoção alguma; as suficientemente suntuosas cadeiras, a cozinha com lava-louças. O carro bom na garagem. Ele teve vontade de jogar a taça de vinho no chão e fazer cara de nojo profundo, como nos filmes. Mas não fez nada. É só uma inquietação, não dê uma de babaca --- pensava.

Tudo começou quando ele era uma criança. Curioso, pa\-re\-ce que tudo sempre começa quando se é criança, mas neste caso nada começou na infância. Maurício nunca teve nenhum problema com a morte. Bom, as crianças não têm muita dimensão dessas coisas mesmo, mas enfim. Medo, talvez; ele temia tudo além de seu entendimento, mas não o suficiente pra não chegar perto. Pra não rondar. Rondar. Essa é a palavra, talvez. Ele está sempre por perto desses mistérios que o assombram. Não é capaz de deixar pra lá, esquecer essas coisas, fingir que não estão ali.

Nunca teve nojo nem horror de saber que somos de carne e osso. Seus olhos brilhavam ao admirar a anatomia humana; não tinha receio de fazer exames de sangue e observava prazerosamente ossos, músculos, sangue. Empolgou-se ao dissecar sapos, ratos, pombos. Sentia frio na espinha quando ficava no escuro; continuava assistindo a filmes de terror, however.

Quis fazer medicina, mas o seu futuro mostrava-se bem distante de suas aspirações mais secretas. Suas aspirações eram brutas e irracionais; um misto de sensações complexas que não se ``acendiam'' com a possibilidade da medicina. Que nobre, que nobre; salvar vidas, curar doentes. Que chato, que desgastante também. Horas, horas, dias de estudo, trabalho mortificante. A medicina mata, dizia um amigo. No mínimo quem quer entendê-la. Ele queria sangue, queria. Trabalhar com anestesias, desfibriladores, bisturis --- mais particularmente esse último. Vivia dizendo que adoraria trabalhar numa casa mortuária, só pra assustar os amigos. Tolinhos.

Nunca foi muito violento. Embora sua expressão carrancuda e seu físico ligeiramente avantajado inspirasse algum medo em quem não o conhecia, ele era amigável e nunca se envolveu com pessoas de fato violentas. ``Porradas'', ``chutes'', ``voadeiras'' --- que coisa mais vulgar. Até armas de fogo lhe pareciam chatas, idiotas, coisa de gente pequena.

Sua grande chance, aparentemente, surgiu enquanto cursava o terceiro ano. Um estágio numa agência de serial killers. Uma das maiores do estado. Compareceu à entrevista; trêmulo, tenso, confuso. É uma grande carreira, é claro; uma vida promissora à frente e um calorzinho em algum lugar ao longo do esôfago o levaram a tentar. Por que não, afinal?

Jamais imaginara que poderia realmente fazer aquilo. É mais ou menos como ser astronauta, ou escritor famoso, ou mesmo músico famoso. É literalmente um universo famoso. São profissões relativamente comuns --- com a exceção dos astronautas --- e que são exercidas por pessoas, pessoas como todo mundo, não por semideuses. É claro que é necessário talento, mas mesmo pessoas talentosas caem na desgraça de não se acharem capazes. É mais ou menos o que acontece com a quase ``astronáutica'' profissão de serial killer.

Poderia ele ser um serial killer? Adolescência é mesmo a época em que se fará tudo que a pessoa já sabia que faria um dia, mas nunca pensou a sério sobre.

A entrevista correu bem, e na outra semana ele foi cha\-ma\-do.

--- Meu nome é Maurício.

--- Oi Maurício, meu nome é Paulo. Bem, hm, bem-vindo! Aquela ali é a Samara, (``fazer piadinha com meu nome é um dedo a menos'') um doce de pessoa. Aquele ali é o Marcos\ldots

E assim Maurício foi com seu chefe conhecer todo o staff. Logo no primeiro dia organizou um bagunçado arquivo de vítimas de 2003. Nada muito interessante, mas tudo bem. Ele teve o prazer de ver Samara chegando quase no fim do expediente, contando como foi legal o trabalho do dia. Ele ficou com um pouco de medo dela, mas no fim acabou rindo, no ônibus de volta pra casa, imaginando as cenas com as informações que pescou nos relatos.

E o tempo foi passando; ele se decidiu pela biologia e passou a ser um empregado permanente da empresa. Fazia sua burocracia com excelência, o que agradava Paulo. Mas não Maurício.

Então um dia Samara pediu as contas. Com menos pessoal, todos passaram a trabalhar mais. Algumas semanas depois, à base de muitas conversas sutis e falsas gentilezas, Maurício foi promovido --- mas não tão rápido; era, agora, aprendiz. Acompanharia os serviços de Fernando.

No campo, com a mão na massa, Maurício foi aprendendo que o trabalho era um pouco diferente daquilo que ele imaginava. Era um pouco tedioso, fosse porque Fernando era curto demais, objetivo demais, fosse porque envolvia, às vezes, uma briga forçada pra provocar o assassinato na forma de legítima defesa. Coisas da lei.

Maurício, carregando e ligando com os corpos e recebendo treinamento técnico de Fernando, chegava em casa exausto depois da faculdade. Dores de cabeça, dores no corpo; não tinha tempo pra mais nada.

Depois de um tempo o dinheiro começou a apertar. Sua faculdade era paga; vivia sozinho mas se mantia com a ajuda dos pais, o que se tornava cada vez mais algo incômodo pra ele. O filho já tinha sonhos, aspirações; queria uma casa pró\-pri\-a, eventualmente um carro próprio, o último iPod, uma jaqueta de couro tão cara quanto a de Marcos, que a usava em assassinatos especiais. Queria, queria, queria. Como e quando conseguiria, se dependia do dinheiro dos pais até mesmo para comer?

Maurício passou a trabalhar independentemente, pela a\-gên\-ci\-a, não só acompanhando outra pessoa. Co\-me\-mo\-ro\-u numa reunião simples com a família, que acabou em briga. O pai pressionava; quando é que ele teria uma mulher, uma companheira, namorada que fosse? A vida tem que ser a\-pro\-ve\-i\-ta\-da, dizia ele. O recém serial killer conhecia algumas garotas interessantes. Teve três ou quatro namoradas, mas nada muito durável. Saiu de lá com gosto de champagne na boca, ainda que amargo. Mesmo sozinho, não chorou. Achava te\-a\-tral demais chorar no ônibus, àquela hora da noite.

O dinheiro a mais que veio com a ascendência profissional o levou a não mais precisar dos pais. Devolveu o depósito feito no mês seguinte e, mesmo não podendo nem chegar perto de um supermercado mais ``convincente'', se sustentou bem.

Matava pessoas com bastante perícia e até mesmo gosto. Era bem detalhista; embora o regulamento não permitisse, ele gostava de um pouco de tortura. Arrancar as unhas, água escaldante aqui e ali. Costumava fazer o serviço com eficiência, em geral.

Logo ele ficou consciente de que seus colegas de trabalho dariam a ele o mesmo tratamento que sempre deram uns aos outros. A competição acirrada, a eventual intriga, a sutil puxada de tapete. É claro que no escritório o tratamento era cordial; eram todos verdadeiros lordes. Mas bastava um trabalho aparecer que as “conexões” certas eram feitas.

Cada um ali tinha um estagiário protegido. Ele preferira, em seus tempos de estágio, manter distância dos matadores e ficar bajulando o chefe. Seja como for, de uma coisa ele estava certo: precisava de um protegido.

Tirando as hostilidades verdadeiras e as cordialidades falsas, Maurício tinha um amigo na empresa. Seu nome era Fábio; era mais velho e já casado. Os dois passaram a sair uma vez ou outra pra beber uma cerveja, ou assistir a jogos de futebol.

Fábio era um cara legal. Mandava e-mails engraçados pra Maurício, com piadas, fotos legais, etc. Maurício se irritava um pouco com isso, mas relevava. E assim ele tinha um amigo, que supria relativamente bem a falta que os antigos faziam. Esses ele via uma vez por semestre ou menos; alguns saíram da cidade, alguns foram pro outro lado da cidade --- o que dava no mesmo, na prática --- mas todos lhe davam os pa\-ra\-béns pelo aniversário na internet. Ele de vez em quando pensava como seria mudar a data de seu aniversário, só pra ver o que acontecia --- mas sempre deixava pra lá, nunca fazia isso.

Então veio Setembro, um mês tumultado. Houve uma gran\-de troca de empregados entre filiais da agência; Fábio foi chamado pra ficar um mês numa agência fora do estado. Vieram alguns serial killers de fora, e a situação ficou ainda mais bagunçada porque o computador da empresa foi atacado e se tornou inutilizável por uma semana.

Naquela semana Paulo perdeu o controle da situação. Um dia Maurício ficou encarregado de matar um homem de meia-idade. Chegando na casa dele, viu Fernando lá, fazendo o serviço. Os dois discutiram, brigaram feio e ameaças foram trocadas. Fernando tinha o péssimo hábito de fazer os serviços que apareciam se ele estivesse por perto, mesmo que não fossem dele. Era uma questão de praticidade, dizia, não importa o quanto Paulo já o tivesse advertido por isso. Segurança o caralho, pensava Maurício. Ele queria era a comissão, e a sua cara de pau era grande o suficiente pra negar isso. Fernando facilmente fazia-se passar por vítima; fazia parecer que a empresa estava errada no modo como gerenciava seus ``recursos humanos''. Psicopata maldito, pensava Maurício. Obviamente que algum estagiário o avisava feliz como um roedor dos infernos sobre o que aparecia lá na empresa. Maldito, filho da puta.

Na outra semana, circulava o boato de que Maurício havia estuprado uma vítima.

Não era verdade; a quase castidade circunstancial de Maurício foi veladamente dita a Fernando há tempos; é o tipo de coisa que se fala usando uma segunda personalidade masculina, assumida quando se está perto somente de outros ho\-mens. Nunca se espera que essas ``cantadas de galo'' ou mesmo ``revelações'' deem em alguma coisa. Dessa vez, deu.

Nada foi provado, o tempo passava, o clima permaneceu tenso --- Maurício foi graduado. Fábio voltou e não se pronunciou sobre o caso. O futebol aos domingos foi ficando cada vez mais raro; é verdade, o campeonato agora era outro, não tinha mais graça. Mas será que era só isso o objetivo das reuniões dos dois?

Mais gente foi contratada, já que a agência crescia vertiginosamente, acompanhando o mercado. Maurício passou a trabalhar nos sábados, sábado sim, sábado não. Mais uma namorada veio e foi. Mais um ano se passou. Ele via no Orkut que algum de seus amigos já se casaram --- um tinha até um filho.

Era engraçado ficar olhando as fotos dos amigos. Parecia um tio mais velho; nossa, como cresceram. Como eles parecem felizes nessas fotos na frente de cachoeiras. Será que são felizes mesmo? E eu, o que é que eu sou? Eu também tenho fotos na frente de cachoeiras. Estão todas no meu Orkut. Já viajei bastante nas minhas férias. É, acho que a minha vida não é ruim não.

Ele começou a trabalhar por fora; juntou-se com um amigo num esquema pra matar clandestinamente e não passar pelo filtro dos impostos. Dinheiro bruto, dinheiro alto; pena que havia pouca procura. Mas uma ou duas noites por mês ele estava lá, matando alguém aparentemente importante. O amigo trabalhava na polícia, e fazia questão de não dar muita atenção a esses casos civis de sonegação, suposta formação de quadrilha e outros\ldots\,

Depois de um tempo pôde comprar um carro melhor; já planejava um apartamento próprio. Apartamento, apesar de seu sonho ser uma casa, porque não podia pagar o preço que cobravam por uma residência decente. Tinha que ser um pequeno apartamento, pelo menos por enquanto.

Um dia ele viu na internet um movimento brasileiro contra os serial killers. Leu um manifesto, viu que muitas pessoas demonstravam seu repúdio contra eles. Propunham o fim das agências; alguns mais moderados propunham uma regulação mais severa. Quando viu aquilo no computador, numa noite fria de quinta-feira chuvosa, ele até tirou o casaco que usava; uma onda de calor passou pelo seu corpo. Ele ficou desnorteado, irritado, tateando na sua mente tudo aquilo que já sabia que justificava tudo o que fazia. Arranjou um par ou mais de silogismos. Procurou no Google e achou algumas opiniões que comparavam essa ideia à de proibir propagandas de cerveja na televisão ou coisa do gênero. Naquela noite não dormiu bem, mas nos dias seguintes estava mais tranquilo.

Dado o teor de alguns comentários que viu na internet, comprou um revólver para segurança pessoal. A paranoia sempre o rondava, e fazia um esforço pra esquecer um pouco o medo que passou a sentir. Logo sentiu medo de ir trabalhar. Comprou um silenciador para o revólver e passou a usá-lo para matar. Em nome da segurança, afinal.

Um ano se passou. Havia comprado uma Sniper --- nem precisava mais ter contato com a vítima. Isso se mostrou pro\-ble\-má\-ti\-co mais tarde, mas ele deu um jeito de se livrar das chatas diretrizes e dos estagiários que as repetiam feito papagaios. Paulo inclusive o apoiava. A Sniper era uma boa ideia, definitivamente.

A Sniper era linda, pensava. Como ele não via antes a beleza das armas. Ele a limpava todas as noites. Com real dedicação. Estava ela lá, rainha absoluta no seu apartamento. Certo que ele guardava o amor à rainha no seu coração, cantando seu hino na imaginação; rainhas não podem se dar ao luxo de ficarem expostas, são parte importante do seu Estado e precisam ser protegidas. Lá estava ela, de baixo da cama. Ele não teve tempo nem ideia para lhe dar um trono digno, mas lá era um bom lugar. Ela era linda, como era. Para ele não havia nada naquele apartamento próprio mais importante do que ela.

Bebeu mais um gole de vinho. Voltara havia meia hora de mais uma de suas aventuras indie. Agradeceu que levou a Sniper consigo, como cada vez mais frequentemente fazia. Ô, decisão acertada. Bom, nada de muito mérito na escolha; ele estava com uma dorzinha nas costas que o fazia ficar até um pouco arcado. Precisou mesmo da Sniper, mas não pensava que ela --- bem, será que ``ela''? --- lhe concederia um bônus naquela noite. E riu\ldots

Era o preço de tanto trabalho, meu Deus. Garantir essa vida, que nada é de graça. A equipe matava cada vez mais. ``Como ainda tem gente pra morrer?'', pensava ele.

E ria. Ele ria.

Lá estava ele. Com seu apartamento mais ou menos. Seu carro mais ou menos. Seu vinho mais ou menos. Seu trabalho mais ou menos. Sua vidinha mais ou menos. Um corpo feminino nu na cama --- morto --- que, pensou ele quando o viu pela mira da sniper --- não era \emph{nada} mais ou menos. E ria. Entre um sorrisão e um sorrisinho, pensava que tudo aconteceria depois do fim, e não antes. Não há nada de errado com isso, afinal.

Era sua filosofia de vida, ele pensava. Temos que rir das coisas ``mais ou menos'' e pensar nas ``pelo menos''. Pelo menos estava vivo. Vivo, mais ou menos. E ria.
