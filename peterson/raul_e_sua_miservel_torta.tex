\chapter{Raul e sua Miserável Torta}

\begin{flushright}
\foreignlanguage{english}{\emph{Not a copy, but a tribute.}}
\end{flushright}

Era uma vez alguém chamado Raul. Não importa quem Raul era; nem mesmo se ainda é. Não importa se é criança, adolescente, adulto, idoso, homem, mulher (o nome pode ser pra esconder a identidade\ldots), branco, preto, alto, baixo, gordo, magro, etc. Importa é que certa vez ofereceram a Raul uma grande e deliciosa torta.

Na verdade, não ofereceram. Nosso personagem misterioso saiu de casa, num belo dia, e então viu a torta ali, no chão, à frente de sua porta. Ele examinou a torta por alguns instantes e pensou ``Bem, vou comê-la!''.

Foi aí que ele percebeu que na embalagem de alumínio --- estilo ``marmita'' --- que envolvia por baixo a torta, havia uma rachadura. Ele olhou desconfiado e pensou ``iiiiii\ldots''. Apesar do ``pé atrás'', ele continuou.

Ele enfim mordeu um pedaço. Estava fria. ``Argh, que droga!'', pensou, ``Pelo menos eu posso esquentar\ldots'', mas ele, apesar de começar a frase com ar de ``Eureka'', a terminou com desânimo. Ficou com preguiça de esquentar a torta. É, talvez ele só precisasse ligar o micro-ondas, mas mesmo assim ele teve preguiça.

Logo depois, uma mulher, de salto-alto, aparentando ser de meia-idade, entrou com altivez no pequeno jardim de sua casa, onde Raulzito permanecia de pé, tentando comer a torta.

--- Você não vai me dar um pedaço?? --- perguntou ela, com inesperada intimidade e em tom de ``é ÓBVIO que você DEVERIA fazer isso, não?''

Raul olhou para os lados, devagar, tentando compreender a situação. Perguntou, o mais educadamente que pôde:

--- É comigo?

--- Eu não gosto de ironias, seu \emph{palhacinho} --- disse ela, séria, ousada, abusivamente confiante --- Você me deve um pedaço. E grande, de preferência\ldots

--- Ei, ei, EI! Espera aí, o quê isso? --- Disse ele, dando um tapa na mão da mulher, que já se aproximava da torta.

--- Como assim, \emph{o que é isso}? \emph{EU} fiz essa torta pra você!

--- \emph{Bom}, Obrigado! --- Respondeu ele, estupefato com a situação ridícula.

--- \emph{Só} obrigado? HÁ!

--- É, não tá bom não, por acaso?

--- Mas eu TROUXE essa torta pra cá.

--- E daí? Problema é seu, ué! Não pedi torta nenhuma! --- respondeu ele, conclusivamente.

--- Mas você gostou, não é? Hein?

--- Bom, é\ldots\,Sim, bem, eu gostei, eu acho --- disse ele, confuso.

--- Então, seu ingrato! Tudo bem, tudo bem, eu admito, posso não ter feito, ou não ter feito toda ela, mas eu trouxe até aqui. Não vai\ldots\,Me dar um\ldots\,Um pedacinho? --- perguntou ela, com ar hesitante nas últimas palavras.

--- Não, é claro que não. Obrigado pela torta, mas não, não vou te dar nenhum pedaço. --- Disse ele, enfim, e a mulher saiu de sua propriedade, triste, porém orgulhosa.

Então, antes que ele pudesse assimilar tudo o que aconteceu, outra mulher entrou em sua casa, dizendo, de modo ainda mais ``ignorante'' (no sentido coloquial de ``brutamontes'', aquele negócio ``animaaaaaaal'', sabe?), que Raul \emph{não tinha} o \emph{direito} de comer aquela torta.

--- Mas por que DIABOS eu não posso comer essa torta?

--- Sabe quantas pessoas passam fome no mundo todos os dias? --- perguntou ela, com um rosto da mais fina indignação.

--- E daí? --- Raul se perguntou, usando as mãos para ajudar a reforçar ideologicamente sua indiferença --- Tudo bem, eu posso ajudá-las, mas não é \emph{passando fome} que eu vou fazer isso!

--- Você é um grande idiota mesmo. Não entende nada! De NADA! --- Disse ela, dando pequenas voltas e fazendo um trajeto irregular pelo jardim. Raul desejou que ela não pisasse tanto na grama --- Você não percebe, é injusto, é injusto que você ou qualquer outra pessoa coma esta torta!

Raul olhou para a torta. Sua torta, fria, em uma embalagem rachada, mas ali estava ela. Um presente esquisito, que ele não pediu. Estava relativamente satisfeito com o fato de que ele não tinha que sacrificar sua torta em prol das ``tantas pessoas'' que passavam fome no mundo. Definitivamente não. Mas mesmo assim, aquela mulher havia conseguido sabotar sua pequena felicidade de comer tortas. De alguma forma aquela torta não parecia mais a mesma. Tinha perdido um pouco o valor.

De repente, um homem veio correndo dos cantos mais longínquos possíveis desse cenário hipotético, parecendo mui\-to cansado. Parou dentro do terreno de Raul e ficou olhando para o chão, ofegante. Exasperado, com as mãos nas coxas, falou:

\begin{sloppypar}
--- Se\ldots\,Você\ldots\,Comer\ldots\,Essa\ldots\,Essa torta\ldots\,O recheio\ldots\ da minha torta\ldots\,\ Vai mudar\ldots\,--- falou ele. Raul não tinha percebido, mas ele trazia uma pequena torta pendurada no pescoço, como num crachá --- ou pelo menos aquilo parecia ser uma torta, bem, dava a entender. Estava dentro de um recipiente bem fechado, pra que não caísse.
\end{sloppypar}

A mulher lançou um olhar de desafio para Raul.

--- Agora você vai ter que reconhecer, você não pode fazer isso!

--- Hã?

--- Você não pode comer essa torta!

--- Por que não?

--- Porque o recheio dele vai mudar, você não ouviu? Largue essa torta \emph{agora mesmo, eu estou avisando}! --- Disse ela, autoritária ao extremo.

--- NÃO! --- Berrou Raul, irritado --- Quer parar com isso? Porra, foda-se, me deixa comer a minha torta em paz! --- Raul entrou em casa. Parou no corredor, e resolveu voltar, ainda achando que talvez não tivesse dito tudo o que tinha pra dizer --- Cacete… Será possível que eu não tenho o direito de comer a minha torta em paz?

Durante algum tempo, silêncio. A mulher parecia irredutível; o homem, cansado.

--- Por favor\ldots\,Senhor\ldots\,Por favor\ldots\,--- Pediu uma última vez o homem.

Raul reconsiderou e segurou a torta com a mão, pra baixo, um pouco desanimado.

--- Tudo bem. Mas eu quero deixar claro que foi porque eu quis! Merda! --- reclamou ele, baixinho, irritado com toda a história maluca.

Depois de alguns minutos olhando em silêncio para os estranhos, ele ficou impaciente de ficar ali, em seu jardim, rodando pra lá e pra cá, sem saber o que fazer ou dizer.

--- Bem, olha, um dia eu vou ter que comer essa torta ou, sei lá, fazer alguma coisa com ela, se não ela vai estragar, né? --- Exclamou Raul.

Então outro homem, muito similar ao primeiro desconhecido que apareceu em sua casa, veio correndo de um outro lugar e disse, dessa vez, com uma voz segura e rápida:

--- Por favor, senhor, não faça isso. Meu recheio mudará consideravelmente pra pior se você comer essa torta. Qualquer pedaço que seja.

Raul olhou estranhamente para a mulher, que parecia estar se divertindo muito com aquilo tudo --- Consideravelmente? --- perguntou o aturdido dono da torta problemática.

--- Sim.

--- Como? Em que sentido?

--- Eu não gosto de ervilhas. Além do mais, não gosto de comer com garfos. --- disse o homem de voz grossa.

Raul riu dos três estranhos ali, uma risada louca, inconsequente, um pouco desiludida, e desistindo de considerar mais absurdo algum, deu uma mordida na torta --- com empolgação.

\sloppy
--- N\-Ã\-Ã\-Ã\-Ã\-Ã\-Ã\-Ã\-Ã\-Ã\-Ã\-Ã\-Ã\-Ã\-Ã\-Ã\-Ã\-Ã\-Ã\-Ã\-Ã\-Ã\-Ã\-ÃÃ\-Ã\-Ã\-Ã\-Ã\-Ã\-Ã\-Ã\-Ã\-Ã\-Ã\-Ã\-Ã\-Ã\-O\-O\-O\-O\-O\-O\-O\-O\-O\-O\-O\-O\-O\-O\-O!!! --- Exclamaram, berraram, gritaram, espernearam, se descabelaram os três estranhos ali no quintal de Raul.

Raul riu ainda mais de si mesmo, dos outros, de tudo. Riu de seu quintal, riu da rua, riu dos visitantes, riu até de sua própria casa, as janelas, a porta, tudo, riu de tudo. Achava aquela cena apocalíptica uma loucura maravilhosa, uma verdadeira libertação. Enquanto o frango desfiado fazia-se sentir nos dentes dele, os homens o chamavam de ``mal'', ``perverso'', ``terrível'', ``lobo'', ``demônio'', entre outros que a imaginação e a memória moral permitiam. A mulher chorava histericamente numa posição estranha no chão.

/bin/bash: :wq: comando não encontrado
Quando terminou de engolir o pedaço que havia pego, Raul olhou ao seu redor pra ver a destruição que havia causado. Todos estavam ali ainda, de alguma forma.

Raul, então, um pouco mais tranquilo quanto aos visitantes malucos, resolveu comer mais um pedaço da torta.

Mas havia algo de errado. A torta não era de carne com ovo antes\ldots\,Era de frango\ldots

Quando ele virou o rosto procurando por algo ou alguém que pudesse explicar o que estivesse acontecendo, viu um homem passando no meio da rua. O passante desconhecido estava de olhos fechados, mastigando tranquilamente um pedaço de torta.

``Droga'' pensou Raul. Ele compreendeu o que aconteceu.

Mas antes que pudesse pensar a respeito, o homem de voz forte que tinha chegado depois já estava na sua frente. Enquanto Raul segurava a torta com a mão direita, ela voltada para cima, o homem tinha se aproveitado de sua distração, talvez, pra começar a colocar pimenta na torta.

--- PUTA QUE O PARIU, O QUE É QUE VOCÊ TÁ FAZENDO, CARALHO? --- Perguntou, surpreendido, Raul.

--- VAI SE FUDEEEEER, AGORA VOCÊ VAI TER O QUE MERECE!

O homem continuava calmamente colocando várias e várias pimentas, de todos os tipos, mas preferencialmente as variações mais fortes possíveis do tempero. O sujeito colocava com prazer as pimentas em cima e dentro da Torta. Ia furando-a com o dedo de forma calma e serena, com um brilho psicopata nos olhos, pra fazer com que o líquido da conserva de pimenta penetrasse em todo o alimento.

O dono da torta assistia a tudo, atônito, sem dizer nada.

Depois que ele terminou, saiu rindo, e então, depois de se distanciar um pouco, parou, apoiado no muro do jardim de Raul, com um rosto satisfeito e suado. Ele parou, olhou para o chão, riu mais uma vez e respirou profundamente, parecendo satisfeito.

--- Desculpe, Raul, mas\ldots\,\emph{foi necessário.} --- Comentou, satisfeita, a mulher, com uma face que poderia ser traduzida pra ``É, não adianta reclamar, você mereceu\ldots''.

--- Droga. Eu queria ter feito isso --- lamentou-se o primeiro homem.

E Raul? Raul continuava ali, com a torta sabotada na mão. Depois as pessoas saíram dali, aos poucos, alguns sem dizer nada, outros dando tchau. E ele voltou pra dentro de casa, pro seu lar verdadeiro. Mas antes, durante muito tempo (ou será pouco? Ele não viu o tempo passar pra medi-lo como sua percepção queria) passou o tempo todo ali, parado, em seu jardim, sentado no chão, olhando para a sua torta. Pobre torta. Fria. Com uma embalagem rachada, estragada pela pimenta, condimento que, em excesso, tornava um alimento intragável --- mas, além disso, Raul não gostava muito de ovo (o novo recheio de sua torta era carne e ovo). Ele teve pena de sua torta, tão frágil, tão destruída, tão judiada. Uma vez depois de um tempo alguém passou lançando um panfleto, dizendo como as tortas eram boas, apesar dos erros de tempero. Alguns segundos depois outra pessoa veio panfletar também, dessa vez fazendo publicidade do dono da fábrica de tortas. O panfleto alardeava que o tempero era perfeito, ideal. Pensou que talvez aquilo fosse besteira. O tempero era pra estar ali, aquela torta é mesmo imperfeita. Talvez todas sejam. Sentiu pena de sua torta. Não quis mais encostar nela, em nenhum pedacinho sequer. Talvez ele pudesse ter recuperado umas partes, talvez. Talvez ele pudesse ter convivido com a pimenta. Ou não. Mas ele nem tentou.

Ele sequer tentou. Ele entrou em casa, e nunca mais foi visto.
