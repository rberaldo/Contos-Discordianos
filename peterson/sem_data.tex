\chapter{Sem Data}

Ele colocou as malas no chão e respirou profundamente. É um novo dia; o começo de uma nova era, praticamente uma revolução em sua vida. Não fosse ali, naquele, diga-se, solo propício, onde mais seria? Um tempo novo. Mas nada de festas pra isso; novo sim, mas comum, como os outros, como um qualquer.

O jovem, lá pela metade dos seus vinte, ficou a inspecionar o apartamento. Era tudo muito bem limpo e cuidado; nada ali cheirava a lembranças, nada ali podia fazer com que sentisse saudades de alguém. E era justamente por isso que tinha se mudado para ali, um lugar onde as memórias não lhe fariam mal.

Apesar de ser vista como soturna e pacata demais, ele achava a vila dos sem-data simpática e aconchegante. A vila era uma ilha de paz e tranquilidade, numa visão geral, em relação ao resto da grande cidade.

Era composta por duas ruas, pequenas, em que as pessoas não se espremiam apenas pela falta de multidão. Os prédios eram de tons claros, e, apesar de bem construídos, pareciam não ser lá muito retos, o que dava um toque todo fantástico ao lugar. Havia algumas barraquinhas aqui e acolá; uma cafeteria, uma loja de cds, um mercadinho. O que aquela vila tinha de tão especial? Ali as datas não são comemoradas. Data alguma.

Muitos órfãos eram mandados para lá. Não havia dia das mães, nem dos pais, a comemorar. Filhos que morriam cedo deixavam pais sem vontade alguma de comemorar o dia das crianças --- e ali moravam tais pais.

Quem conhecia a vila, seja porque alugava lugares lá ou por carinho que tinha com os moradores, poderia sempre enumerar casos e casos específicos. Gente que tinha raiva do natal, da páscoa. Parentes de pessoas torturadas e mortas por militares durante a ditadura que não suportavam o dia da bandeira; a lista era extensa, e muito variável.

O novo morador da vila não tinha um caso tão complexo ou trágico. Ele só queria distância do dia dos namorados; logo que pensou nisso seus pensamentos voaram, suaves, silenciosos, tortuosos em direção a ela; linda. Linda, linda e tão, tão falsa. Tão insensível. Seria a vila dos sem-data a vila da tortura? Uma maldição disfarçada, que faz, por vontade de esquecer, lembrar? Seria lembrar, lembrar até se acabar --- lembrar até se cansar --- o caminho pra nunca mais querer ver pela frente o próprio passado?

Ele chorou ao pensar assim, em espiral, cada vez mais negativamente\ldots Chorar é quase sempre explodir. A construção da bomba se dá pouco a pouco, numa progressão --- de constatações ou calúnias que vêm de pouco autoestima --- que culminam numa grande explosão, que dá pra sentir naquele ápice de miséria, tão característico do começo teatral de um bom choro.

Ela não deveria ter feito isso, não poderia ter feito isso, pensava ele. ''A quem eu queria enganar, vindo pra cá''?

Mas não havia ninguém pra enganar. Com intenção, sem intenção --- ali não havia dissimulações. Estava estampado no rosto de todo mundo. Estava no rosto do padeiro. Estava naquela mulher de cabelo curto que ''olha o movimento'' pela janela todas as tardes, como estátua. Está na simpática senhora do mercadinho. Todos ali sabiam da verdade:

Se você está aqui, é porque você sofreu.

Ele enxugou o rosto do melhor jeito que pôde com as mãos, e nenhuma outra lágrima caiu. Ele olhou em volta. Foi até o banheiro, abriu a torneira, viu que havia uma garrafa de água na geladeira. Ele ia viver ali. Seria bom. Quem sabe, quem sabe; ele poderia até ser feliz.
