\chapter{Ao Teu com o Diabo}


\begin{center}
{\Large I}
\end{center}

``Não existe ateu em avião em queda''. É o que não cansam de repetir os crentes e eu, em meu agnosticismo de ateu, nunca dei crédito à premissa.

Eis que, rumo a Madrid para um simpósio, meu avião entra em turbulência.

Não pense você que, por não acreditar em Deus, tenha na ciência o depósito de minha fé. Em parte sim, é verdade. De todo modo, não vejo muita distinção entre os fanáticos religiosos e alguns cientistas, em especial os físicos, que nunca vi povinho mais dado a acreditar em teorias completamente lunáticas.

Posto isso, não é exagero dizer que tenho medo de voar, até porque o que sinto é, senão, verdadeiro pavor! Tremi todo junto com a aeronave e recostei-me na poltrona com os olhos fechados. Logo que a situação se regularizou, as comissárias correram aos seus carrinhos para servir os lanches, antes que não fosse mais possível.

Conhecendo meu sistema digestório sensível e já prevendo uma próxima instabilidade, dispensei meu lanche, servindo-me apenas de um pouco de álcool. Recostei-me novamente, de olhos bem abertos dessa vez, e comecei a distrair-me com pensamentos aleatórios.

Acabei entrando na questão de meu ateísmo; se não era o caso de eu ser agnóstico, talvez\ldots\,Estava a ponto de puxar conversa com o agnóstico sentado ao meu lado --- já o conhecia de uma outra conferência qualquer, mas por não saber seu nome e não querer interromper a degustação que fazia de sua xícara de chá, vacilei, analisando a melhor forma de abordá-lo --- quando entramos em uma nova turbulência.

Mais uma vez me subiu um frio na espinha e, não tendo o avião se reestabilizado relativamente rápido --- como ocorreu na anterior --- o pânico começou a se instaurar no interior da cabine. Ao tentar desviar meu olhar do desespero alheio, acabei por mirar a janelinha e a tempestade que se dava lá fora. Baixei com violência a persiana, olhando para o teto e exclamando um ``ai meu Deus!'', vício de linguagem que eu havia há muito apagado do meu vocabulário.

Passado o choque de meu apelo inconsciente, o pensamento nefasto de que apenas uma força espiritual me salvaria tomou conta de mim. Nesse momento percebi a diferença nada sutil entre o agnóstico e o ateu (imagine que, em meio àquele caos, meu colega continuava concentrado em seu chá, como se nada estivesse acontecendo ao seu redor).

Assim que admiti para mim mesmo que minha única salvação seria divina, comecei a pensar em para quem rezar. Não poderia abrir mão de comer carne, tampouco colaborar com alguém que manda seus fiéis apedrejar mulheres. Minha mãe não era judia e o panteão latino era taylorista por demais, nunca saberia para quem apelar nessa circunstância.

Restou-me o deus cristão, mas tinha lá minhas dúvidas de que Ele era de fato ``só amor''. Durante minha vida, nunca fui lá muito simpático com Sua omnissapiência e não O imaginava muito inclinado a perdoar minhas bastantes heresias a tempo de salvar-me do acidente iminente.

Num flash de razão, lembrei que admitir a existência de um Deus era aceitar igualmente sua antítese. Depois de uma breve entrevista com minha consciência, decidi vender minha alma imortal ao Diabo, ao preço da salvação de minha vida terrena.

Dos males, o menor. Sei lá se por isso ou não, a voz do piloto logo soou, avisando que já havíamos voltado ao normal e pousaríamos em Madrid dentro de vinte minutos.

\newpage
\begin{center}
{\Large II}
\end{center}

Engraçado que sempre considerei os agnósticos ``ateus fracos'' e acreditei piamente que o ceticismo radical fosse muito mais ``seguro'', devido seu caráter estável-estático.

Quem sabe seja exatamente nessa estabilidade, na solidez ideológica, que esteja o ``perigo''. Confesso não ser o maior entendedor das questões físicas, mas não é verdade que as edificações precisam de um certo balanço pra que não sejam derrubadas pelo vento ou tremores de terra? Por que seria tão diferente conosco, humanos, a ponto de considerarmos os flexíveis fracos e jogarmo-los no balaio dos indecisos e irrelevantes?

Sei que nesse meio tempo, em que me entretinha com tais questões, minha mala deu algumas voltas na esteira (vi de relance um volume amarelo passando e descarto a possibilidade de outra pessoa no mesmo voo ter uma mala de viagem amarela).

Não digo que me conformei, apenas que mais uma vez as futilidades do mundo externo acabaram suprimindo minha abstração.

Retirei por fim a tal mala amarela da esteira e rumei para o saguão. Um rapaz qualquer segurava preguiçosamente uma plaquinha precária, improvisada de um pedaço de papelão em que meu nome fora escrito com caligrafia primária, um ``T'' a menos, um ``C'' a mais e alguns respingos de gordura que eu não cheguei a ver, mas pude perfeitamente imaginar.

Bem, não havia muito que pudesse ser feito. Cheguei-me a ele, impaciente para ser levado de uma vez ao hotel, torcendo para que fosse daqueles condutores silenciosos e rápidos. Nenhum-nem outro; desatou a falar comigo.

Primeiramente, disse que não sabia quem eu era. Óbvio que não; não sou o tipo de pesquisador que recorre à mídia de massas. Tenho alguma dignidade --- arrogância academicista, como queira. Então\ldots

``Comecei a pesquisar. Não é todo dia que\ldots\,Na verdade é mais comum do que você deve imaginar, mas enfim. Tinha que saber alguma coisa sobre quem eu estava levando. E devo dizer que foi um ótimo negócio. Um pesquisador tão importante quanto você\ldots\,O que paguei por sua alma foi uma niñeria!''

Talvez porque nesse momento um carro nos fechou no cruzamento e começaram a soar buzinas de todos os lados, acabei alucinando as últimas sentenças dele. Fiquei meio assim de pedir que repetisse, pareceu um pouco deseducado de minha parte. Mi culpa; não prestava mesmo muita atenção já que o assunto dele não me era interessante.

Só para não dizer que pouco me importava, assim que deixamos o furdúncio do cruzamento tomei os cuidados de perguntar seu nome.

	\vspace{3em}

	\begin{center}
	\emph{Os movimentos finais de ``Ao Teu com o Diabo'' ainda dormem desconhecidos na mente turbulenta da jovem Carol Peters.}
	\end{center}
