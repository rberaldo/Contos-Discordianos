\chapter{Introdução}

%Gostosamente o Discordiano Original descia a ladeira, slakdosamente, compondo poemas em louvor dA Deusa. Bardosas rimas escorriam em rios de hidromel; deuses, gigantes, traficantes, banqueiros, samurais, filósofos e astrônomos numa memorável guerra pelejavam.

%Oh, mas, ah!, deixem-me contar que este Discordiano sou eu. Escutem, homens de fé, a história qu'A Deusa me deu.

%\begin{verse}
%Éfe, ésse, éle, vê.
%
%Ó homens de fé, que informados estão \\
%Das peripécias dA Deusa. \\
%Eis aquilo que aconteceu ao aedos \\
%Gregos, difamadores d'Éris!
%
%Pestilência, fome e desgraça! \\
%Prepotência, inveja e pobreza! \\
%Eis o destino dos filhos d'Homero \\
%Que como o pai desonraram as letras! 
%
%Pintores, artistas, escritores, \\
%Atores, músicos e'scultores, \\
%As musas esqueçam! \\
%Há apenas uma Deusa,
%
%E Ela é a Sua Deusa. \\
%Todos aqueles que ignorarem \\
%Este Ensinamento, \\
%Estão condenados ao esquecimento.
%
%Eis o terrível destino, \\
%selado por Éris que, sorrindo, \\
%Promete agora nova chance aos \\
%Aventureiros das letras.
%
%Skaldosamente bardosas rimas \\
%Me fojem entre os dedos, e a pena \\
%Se recusa a contar mais. \\
%Eis qu'este livro de contos,
%
%Abençoado pelA Deusa, \\
%Escrito por muitos, \\
%Introduzido está, \\
%Pelo Bardo Verde.
%\end{verse}

A escrever.
