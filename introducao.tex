\newpage
\chapter*{Introdução}

Schopenhauer uma vez disse: ler significa pensar com a cabeça alheia em vez de pensar com a própria. Então permita-se pensar com a minha cabeça por alguns instantes. Visitar a história do discordianismo, como uma forma de entender por que nós estamos aqui e por que você vai se tornar mais um discordiano (aqui eu dou uma risada diabólica). Representante moderno de Carneades, o discordianismo é uma religião baseada no caos no ano de 1958. Qualquer afirmação sobre o discordianismo, nunca sobrevive a um exame mais minucioso. Isso por que, divergir sobre o que são e o que fazem, é lei entre os que se declaram praticantes do mesmo. Primeiro, porque para alguns o discordianismo é apenas uma sátira, uma piada disfarçada de religião. Para outros, na verdade é uma religião disfarçada de piada.
 
\begin{flushleft}
{\Large \textbf{Criação}}
\end{flushleft}


Os criadores do discordianismo foram Gregor Hill, também conhecido como Malaclypse the Younger, é o autor do principal livro, o \emph{Principia Discordia} e Kerry Wendell Thornley, ou Omar Khayyam Ravenhurst ou ainda Ho Chi Zen.E foi desenvolvido como um exercício de guerrilha ontológica no que eles chamavam de Operação:~Mindfuck através da ``versão Irmão Marx do zen'', o discordianismo.

\begin{flushleft}
{\Large \textbf{Zen}}
\end{flushleft}

O discordianismo embora a primeira vista não pareça, é o zen ocidental. Kerry Thornley, anos mais tarde de criar o discordianismo, sob o nome de Ho Chi Zen, lançou uma série de panfletos sobre a zenarquia. O zen nasceu na China, como uma escola do budismo mahayana, que é notável por sua ênfase na plena aceitação do momento presente, ação espontânea, e o abandono do pensamento julgamentoso e auto consciente. O zen ainda se divide em vários ramos, sendo mais notórios dois deles: Soto e Rinzai. Enquanto a escola Soto dá maior ênfase à meditação silenciosa, a escola Rinzai faz amplo uso dos koans.

Koans são histórias, diálogos, questões, ou afirmações geralmente contendo aspectos que são inacessíveis ao pensamento racional, ainda que possam ser acessados à intuição. Um dos mais famosos e que figura no \emph{Principia Discordia} é este: ``Qual o som de palmas com uma mão só?'' Da mesma forma, o discordianismo faz amplo uso de histórias, diálogos, questões, afirmações, imagens e qualquer coisa que provoque a confusão, a Operação:~Mindfuck. O propósito é sacudir as pessoas de suas zonas de conforto e levá-las a pensar.

\begin{flushleft}
{\Large \textbf{Caos}}
\end{flushleft}

Os discordianos que seguem o erisianismo, usam Éris, a deusa grega da discórdia como divindade. A palavra caos irá aparecer muitas vezes no material discordiano. Sobre tal é digno de nota, que para um discordiano, caos não é antônimo de ordem. Para eles, o caos é a natureza da realidade. O antônimo de ordem é a desordem. Eles apenas querem conscientizar a sociedade moderna que busca a ordem em tudo, de que vivemos em um Universo caótico e que não existe essa coisa que chamamos ``verdade''. Como escreveu Robert Anton Wilson, também conhecido como Dr.~Mordecai Malignatius no meio discordiano:

A iluminação discordiana é alcançada quando você se conscientiza de que, apesar de a deusa Éris e de a lei dos cinco não serem literalmente verdadeiras, nada é literalmente verdadeiro. Dos cem milhões de sinais zunindo, recebidos a cada minuto, o cérebro humano ignora a maioria organiza o resto em conformidade com qualquer sistema de crença estabelecido nele. Podemos selecionar sinais ordeiros e legais e dizer que tudo é projetado por uma inteligência cósmica, como no tomismo, ou selecionar sinais caóticos e afirmar que Deus é uma Mulher Louca, como no discordianismo. O cérebro ajustará os sinais recebidos aos dois sistemas de crença…ou a uma dúzia de outros.
 
\begin{flushleft}
{\Large \textbf{Brasil}}
\end{flushleft}

Não é certo quando o discordianismo chegou ao Brasil. Mas foi graças à sua presença na internet que ele conseguiu continuar existindo em nossas terras. Nos últimos anos, com o avanço de algumas descobertas em psicologia e física se aproximando dos ideais pregados pelo discordianismo e o aumento de sua presença na internet, ocorreu um aumento no número de seguidores.  Mantinha-se comunidades em redes sociais com gatos pingados aqui e ali discutindo a filosofia enquanto digitavam ``fnord'' aqui e ali ou discutiam se o gato de schrödinger estava vivo ou morto. Então, foi traduzido o \emph{Principia Discordia} em português por um sujeito franzino, egoísta, megalomaníaco e que gosta de falar de si mesmo na terceira pessoa. Atropelado por uma moto enquanto ia para seu trabalho de bicicleta ganhou um afastamento médico de quinze dias. Sem nada para fazer a não ser ler, resolveu traduzir o livro que considerava o ápice da espécie humana. Pegou um caderno com a capa do Bob Esponja e se lançou à tarefa da tradução.

Pouco tempo depois ele iniciou um blog que se declarava ser uma Cabala Discordiana. A Cabala 1001~Gatos de Schrödinger atraiu religiosos raivosos, adolescentes na puberdade, artistas, políticos, pedófilos, pessoas grosseiras e um pequeno grupos de pessoas que mais tarde, sem embaraço nenhum, iriam se declarar discordianos. E iniciou-se um pequeno movimento que gerou a criação de cabelas e a criação de uma rede social entre os mesmos.

Os contos aqui reunidos são de pessoas que de alguma forma tomaram contato com o \emph{Principia Discordia} e que de uma forma ou outra sentaram suas bundas em suas cadeiras e escreveram coisas fodásticamente incríveis ou ridículas, deixo o juízo estético para cada um de vocês em especial Kant. Não é interessante como essa coisa funciona? Num dia um idiota qualquer é atropelado, coloca-se a traduzir um livro escrito por um bando de loucos americanos que não tinham qualquer objetivo a não ser tirar um sarro de todo o Universo e faz com que um bando de garotos selvagens e avatares de Éris escrevam coisas mais ou menos relacionadas com tudo aquilo. Do pouco que eu conheço do idiota em questão acredito que ele esteja orgulhoso.

Felipe Schneider, o sujeito que abre este livro, me pegou de jeito, o sacana. Prende a leitura. Carol Peters é puro ouro. Ou melhor, uma maçã de ouro puro. Sobre o Rev.~Rafael Beraldo eu não posso dizer nada pois como ele quem é o diagramador tenho medo do que o bastardo possa fazer, colocar palavras na minha boca só para me difamar por exemplo. O Duubhglas Juarezzz é tipo a nossa Hebe Camargo, discordiano \foreignlanguage{english}{\emph{old school}}. E escreve incrivelmente. Já o Rev.~Peterson é o nosso Kaká. Jovem e craque. Aliás, já que nosso país tem uma relação com o futebol muito forte e as metáforas relacionadas ao mesmo sempre veem a calhar, com este time nós ganharíamos a Copa. Diversas vezes. Pentacampeão com certeza.

Logo, a história do discordianismo no Brasil passa por aqui. Por estas palavras. E vai continuar nas palavras dos contos a seguir. E continuará, quem sabe, nas ações e palavras suas. Sim, estou falando com você. Se você nunca se imaginou sendo discordiano um dia, é tarde demais. Nós já pegamos você.

Qual o parasita mais resiliente? Um vírus? Uma bactéria? Um verme intestinal? Não. Uma ideia. Uma ideia pode reescrever todas as regras e mudar o mundo. E a ideia do discordianismo continua por aí, pegando algumas mentes-que-realmente-pensam desprevenidos e os convertendo instantaneamente. O discordianismo não faz prisioneiros. Mas definir o que ele é? Esqueça. Seja ele uma piada disfarçada de religião ou uma religião disfarçada de piada, parece que o discordianismo veio para ficar.

	\begin{flushright}
	\emph{Ibrahim Cesar,\\
	Primavera de 3176 \textsc{yold}}
	\end{flushright}
