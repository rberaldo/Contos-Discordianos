\chapter{O cachorro de Pavlov e o gato de Schrödinger}

\begin{center}
{\Large I}
\end{center}

Pééé!

Esse era o som da campainha. É, eu sei\ldots\,Uma grande bosta, mas com o tempo você se acostuma. O tempo te deixa suscetível a qualquer merda. E depois de um tempo, você até sente falta quando viaja.

Mas, quando se é um cachorro, você não viaja muito. E se você é um cachorro inteligente que ouve rádio-novela, você descobre que em algumas viagens os cachorros não voltam. Mas isso não impede que você abane a porra do rabo toda vez que vê o dono balançar a chave do carro.

Se você é um cachorro inteligente de verdade, você não fica ouvindo rádio-novela.

Mas não tem muita coisa pra se fazer nesse ano de 1931.

Aliás, você é inteligente de verdade? Você está lendo um texto escrito por um cachorro!

Eu sou um cachorro russo, nascido em 1931.

E meu ex-dono era um russo com fetiches por cachorros.

Meu ex-dono era um russo com fetiches por salivação de cachorros.

Agora, eu vivo com um austríaco que gosta de manter gatos dentro de caixas.

Pelo menos eu não preciso mais ter vergonha de salivar.

\begin{center}
{\Large II}
\end{center}

Uma das vantagens de ser cachorro é que você é idiota demais pra saber de antemão que você vai morrer. Humanos não.

Humanos se assustam com a possibilidade de morrer.

Alguns dizem que os animais sabem instintivamente do ciclo da natureza e simplesmente não temem a morte porque ela é algo natural. Eu sou um cachorro, e digo que se soubesse quando eu vou morrer, eu choraria feito um recém nascido.

Sim, sim\ldots\,Paradoxal, não?

Humanos se assustam com a possibilidade de morrer e ainda assim têm esperança em depois da morte.

Se eu soubesse quando eu vou morrer, eu choraria feito um recém nascido.

Quando você sai do útero, é como a morte. Você não faz ideia de para onde está indo.

Imagine irmãos gêmeos, com sua linguagem de feto, num papo altamente filosófico:

``Você acredita que existe vida depois do útero?''

``Isso é besteira, rapaz\ldots\,Depois daqui, acabou.''

``Eu gosto de acreditar que tem alguma coisa\ldots''

``Ah, isso é para os fracos\ldots''

Depois do útero, acredito que o pensamento seria algo tipo:

``Merda, o útero era tão bom\ldots\,Por que tenho que ser condenado a morar do lado de fora?''

E durante a vida, as pessoas conversam se existe vida depois da morte\ldots\,E essa crença faz muitas pessoas terem vontade de viver.

Sim, sim\ldots\,Paradoxal, não?

\begin{sloppypar}
Do útero pra esse mundo\ldots\,er\ldots\,bizarro! Uma grande queda de qualidade não? E ainda assim, esperam muito do que vem depois\ldots
\end{sloppypar}

Acho que as pessoas depositam muita fé na morte.

Quando você sabe que vai morrer, seu coração amolece. Você quer consertar as merdas que fez durante a vida.

Mas não todas, só as que te atormentam.

Aquelas que você teve orgulho demais pra se desculpar enquanto tinha tempo hábil pra isso. Aquelas que te acordam com pesadelos de noite. Que te fazem ter medo de escuro, quando você acha que não tem medo de nada.

Mas você sente que sua missão é, de alguma forma, tentar compensar.

O que não parece ser muito justo, pois uma boa ação não perdoa uma má ação. Ela só melhora o julgamento das pessoas sobre você.

Quando melhora\ldots

Mas o que conta são as aparências, não?

De qualquer maneira, você quer fazer algo pra compensar.

Mas pra quê? Pra limpar o nome no ``livro da vida''? Pra não pagar impostos no jogo cósmico?

Por que, se não, você vai para um lugar ruim depois da morte?

Se a lógica é essa, antes de ir parar num útero, todos os que estão nesse universo devem ter cometido atrocidades vergonhosas para ter vindo parar aqui.

Só uma pequena observação: ``atrocidades vergonhosas'' pode parecer pleonasmo, mas não é. É uma questão de julgamento.

\begin{center}
{\Large III}
\end{center}

Todo dia, o barbudo trazia meu alimento, e em vez de me entregar de uma vez, gostava de me ver salivar, esperando ansiosamente aquele pedaço de pão.

\enlargethispage{\baselineskip}

Russos não sabem alimentar cães muito bem, presumo eu. Eu não tenho muita experiência em alimentar cachorros, só em ser um cachorro alimentado.

Pão não é grandes coisas\ldots\,Mas é melhor que nada.

É suficiente para que eu salive.

Digo, hoje em dia.

No começo, eu não salivava. Achava absurdo alimentar cães com pães.

``Cães com pães''. ``Cães com pães''. ``Cães com pães''. Legal, não? Vai ver foi num trocadilho desses que surgiu o ca\-chor\-ro-quen\-te.

Como eu disse, eu achava absurdo ver aquele monte de cães salivando por pão. Me recusava a ceder para um fetiche por saliva canina de um russo barbudo.

Mas comia o pão. Era tudo que eu tinha pra comer.

E o tempo foi passando, e o tempo é uma droga que você tem que tomar mesmo sem que te receitem, e ele te deixa suscetível a qualquer merda.

O tempo te deixa suscetível a qualquer merda mesmo.

Mesmo.

Até salivar por pão.

Um cão salivar por pão!

(Sem piadinhas dessa vez, desculpe. Eu tenho senso de ridículo --- infelizmente)

\enlargethispage{\baselineskip}

Acabei cedendo, para a felicidade do russo barbudo.

Em 1935, a saúde do barbudo já não era a mesma. E o prenúncio da morte veio com pacote completo. E isso inclui arrependimento.

E ele decidiu dar a todos os seus cães uma vida melhor, com esperança.

Mas o barbudo era velho, e ele sabia que acabar com um hábito tão importante poderia abreviar ainda mais sua vida.

Mas o barbudo era velho, e ele estava desesperado por salvação.

Coitado.

O russo barbudo decidiu escolher um cachorro para ser salvo. Um só.

Salvo\ldots\,Rá!

Salvo dele.

Quem, nesse universo, está a salvo de verdade?

Alguns dos cachorros eram muito idiotas para pensar, e ele gostava disso. O russo barbudo com fetiche por saliva de cachorro gostava dos cachorros que atendiam suas expectativas sem questionar nada.

Alguns cachorros eram espertos demais para pensar, e ele gostava disso. O russo barbudo com fetiche por saliva de cachorro gostava dos cachorros que atendiam suas expectativas sem questionar nada.

Ele tinha que escolher um cachorro mediano. Ele tinha que escolher um nem-que-sim, nem-que-não.

Ele tinha que me escolher.

Ele tinha que me escolher\ldots

Ele tinha que me escolher?! LOGO EU?!

\newpage
\begin{center}
{\Large IV}
\end{center}

Um russo barbudo com fetiche por saliva canina jamais admitiria publicamente seu fetiche. Ele sempre achou que ninguém aceitaria. Jamais tentou contar pra ninguém seu fetiche.

Russo. Barbudo. Fetiche por saliva. Saliva canina.

Só uma pequena observação: Russo barbudo não é pleonasmo. É clichê.

Aposto que, se o russo procurasse, ele conseguiria achar alguém que gostasse de observar cães salivando.

Aposto que se o russo achasse, ele teria nojo dessa pessoa. Aposto que o russo tinha nojo de si mesmo.

Como todo mundo, TODO MUNDO, o russo tem nojo de algum aspecto em si próprio.

Aposto que se o russo procurasse, ele talvez encontrasse um amigo que gostasse de observar cachorros salivando e que não tem nojo de si próprio. E seria um amigo canino.

Cães não tem muito nojo\ldots\,Vai ver é por isso que os cães são os melhores amigos do homem.

O russo barbudo nunca conseguiu se aproximar de ninguém, pois não conseguia manter uma amizade guardando esse segredo. Não conseguia criar vínculo. Isso o afastava das pessoas.

Como todo mundo, TODO MUNDO, o russo foi mesquinho, e maquiou sua própria perversão.

Para o mundo, ele não tinha fetiche por observar cães salivando, ele ESTUDAVA comportamento canino. Aliás, ele pensou melhor ainda: ele estudava comportamento.

Não faz a menor diferença para mim ou qualquer dos outros cachorros.

Não faz a menor diferença pra ele. Ele ainda sente nojo de si próprio, e ele ainda continua com seu fetiche.

Pessoas com o vícios parecidos têm o costume de criar grupos de apoio para sustentar seus vícios. O russo barbudo não tinha amigos, mas algumas pessoas consideravam seu hábito de observar cachorros salivando como uma ciência séria.

Numa dessas reuniões de cientistas sérios, em 1935, em algum lugar da Europa, o russo barbudo conheceu um austríaco maníaco.

``Austríaco maníaco''. ``Austríaco maníaco''. ``Austríaco maníaco''. Soa engraçado.

Então\ldots\,O austríaco maníaco tinha fetiche por gatos trancafiados.

Gatos trancafiados.

Isso mesmo. E não é tudo.

Ele tinha fetiche por trancar gatos em caixas seladas e com armadilhas dentro.

E observar.

Coitado do gato.

Mas pelo menos, ele me alimentava muito bem.

Eu já havia esquecido dos meus parentes comendo pão. E muito provavelmente, a consciência do russo barbudo estava limpa.

Missão cumprida.

\begin{center}
{\Large V}
\end{center}

Depois de alguns meses de alimentação saudável, passei a me parecer mais com um cachorro, fazer mais coisas de cachorro.

Alimentação saudável quer dizer que eu não estou comendo mais pão. Não que o que eu esteja comendo seja uma maravilha, mas também não é uma porcaria, tipo\ldots\,pão.

Com meus instintos caninos de volta, decidi fazer o que qualquer cão são faria desde o primeiro dia na casa do austríaco maníaco.

``Cão são''. ``Austríaco maníaco''. ``Cão são''. ``Austríaco maníaco''. ``Cão são''. ``Austríaco maníaco''. Dessa vez foi pra não perder o costume.

Então\ldots\,Decidi atacar um gato.

Como é que eu não tinha pensado nisso antes?

Ah, sim\ldots\,Lembrei\ldots

Estava muito ocupado perseguindo pães.

Alguns hábitos demoram a serem esquecidos\ldots\,Essa maldita mania por segurança.

Então\ldots\,Gatos. Decidi atacar um gato.

O austríaco maníaco tinha muitos gatos. E muitas caixas. E muitas armadilhas.

Um gato específico me chamou atenção. Não o mais idiota, nem o mais esperto. Um gato mediano.

Um dia, um desses ``cientistas sérios'' veio visitar o austríaco maníaco. E ouvi a conversa dos dois.

O austríaco maníaco diz que tranca gatos em caixas com uma armadilha especialmente preparada, que tem 50\% de chance de disparar.

O austríaco maníaco diz que, enquanto a caixa está trancada, o gato está 50\% vivo e 50\% morto, devido às probabilidades. Só quando se abre a caixa e se constata o estado de saúde do gato, uma das probabilidades, vivo ou morto, se projeta. Antes, o gato estava tanto vivo quanto morto. Ao mesmo tempo.

É o cara ou coroa mais louco que eu já vi.

O austríaco maníaco geralmente dava palestras sobre gatos em caixas com armadilhas para públicos grandes. Mas dessa vez era diferente, alguém havia o procurado em sua residência empolgado com a ideia.

Ele tinha que demonstrar.

Ele tinha que demonstrar com algum gato.

Ele tinha que demonstrar com um gato que não fosse o mais idiota e nem o mais esperto. Um gato que não fizesse falta. Ele tinha que demonstrar com um gato medíocre.

Ele tinha que demonstrar com o gato que eu havia escolhido atacar.

Ele tinha que demonstrar com o gato que eu havia escolhido atacar\ldots

Ele tinha que demonstrar com o gato que eu havia escolhido atacar?! LOGO ELE?!

Se o gato fosse pra caixa, ele estaria protegido.

Protegido de mim\ldots\,Mas quem ia proteger o gato da caixa?

E eu queria atacar o gato. Eu queria matar o gato. Era essencial pra minha reabilitação como cão.

E com o gato na caixa, por alguns instantes, ele estaria vivo e morto. Ao mesmo tempo.

Mas eu queria atacar o gato. Tirar a vida do gato. Matar o gato. Assassinar o gato.

O austríaco maníaco pegou uma caixa. O austríaco maníaco pegou o gato. O que eu havia escolhido. O austríaco maníaco pegou uma armadilha.

Nessa hora, tomei uma decisão.

Antes do austríaco maníaco selar a caixa, sem que ele percebesse, entrei na caixa.

O austríaco maníaco selou a caixa.

Selando a caixa, a armadilha estava preparada.

Por alguns instantes, eu estive vivo e morto. Ao mesmo tempo.

E fiquei confuso se deveria matar o gato ou não.

Aliás, eu não sabia se eu mesmo estava vivo ou morto. E eu não sabia se o gato estava vivo ou morto.

Se eu estivesse morto, como eu iria matar o gato?

E se o gato já estivesse morto, pra que matar o gato?

Por garantia, decidi morder o gato.

Do meu ponto de vista, ele parece 100\% morto.

Mas o austríaco maníaco sabe mais sobre gatos em caixas que eu, e há mais tempo que eu. Então, o gato está 50\% vivo e 50\% morto ao mesmo tempo.

E então, a armadilha dispara.

A armadilha disparou.

A armadilha que era pro gato, disparou. E acertou em mim.

A armadilha que era pro gato acertou em mim.

Eu morri.

Morri.

Eu morri. Morri.

Morri.

Estranho, mas eu estou 100\% morto.

100\% morto.

Merda!

Acho que a armadilha deixar 50\% vivo e 50\% morto ao mesmo tempo só deve funcionar com gatos.

Merda\ldots\,Queria saber se, depois de a caixa ser aberta, o gato vai ficar 100\% vivo ou 100\% morto.

Merda!

Merda\ldots\,Queria ver a cara do austríaco maníaco quando abrir a caixa.

Merda!
